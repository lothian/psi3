%
% The PSI Installation Manual
%

\documentclass[12pt]{article}
\usepackage{html}
\setlength{\textheight}{9in}
\setlength{\textwidth}{6.5in}
\setlength{\hoffset}{0in}
\setlength{\voffset}{0in}
\setlength{\headheight}{0in}
\setlength{\headsep}{0in}
\setlength{\topmargin}{0in}
\setlength{\oddsidemargin}{-0.05in}
\setlength{\evensidemargin}{-0.05in}
\setlength{\marginparsep}{0in}
\setlength{\marginparwidth}{0in}
\setlength{\parsep}{0.8ex}
\setlength{\parskip}{1ex plus \fill}
\baselineskip 18pt
\renewcommand{\topfraction}{.8}
\renewcommand{\bottomfraction}{.2}

\begin{document}

\newcommand{\PSItwo}{{\tt PSI2}}
\newcommand{\PSIthree}{{\tt PSI3}}

%
% Psi Modules
%
\def\module#1{{\tt #1}}
\newcommand{\PSIdriver}{\module{psi}}
\newcommand{\PSIinput}{\module{input}}
\newcommand{\PSIcints}{\module{cints}}
\newcommand{\PSIcderiv}{\module{cints --deriv1}}
\newcommand{\PSIdetci}{\module{detci}}
\newcommand{\PSIdetcas}{\module{detcas}}
\newcommand{\PSIdetcasman}{\module{detcasman}}
\newcommand{\PSIclag}{\module{clag}}
\newcommand{\PSIccenergy}{\module{ccenergy}}
\newcommand{\PSIccsort}{\module{ccsort}}
\newcommand{\PSIpsi}{\module{psi}}
\newcommand{\PSIcscf}{\module{cscf}}
\newcommand{\PSIoptking}{\module{optking}}
\newcommand{\PSItransqt}{\module{transqt}}
\newcommand{\PSInormco}{\module{normco}}
\newcommand{\PSIintder}{\module{intder95}}
\newcommand{\PSIgeom}{\module{geom}}
\newcommand{\PSIoeprop}{\module{oeprop}}

%
% Psi Library
%
\def\library#1{{\tt #1}}

%
% Psi and Unix Files
%
\def\FILE#1{{\tt file#1}}
\def\file#1{{\tt #1}}
\newcommand{\inputdat}{\file{input.dat}}
\newcommand{\outputdat}{\file{output.dat}}
\newcommand{\fconstdat}{\file{fconst.dat}}
\newcommand{\intcodat}{\file{intco.dat}}
\newcommand{\optaux}{\file{opt.aux}}
\newcommand{\basisdat}{\file{basis.dat}}
\newcommand{\pbasisdat}{\file{pbasis.dat}}
\newcommand{\geomdat}{\file{geom.dat}}
\newcommand{\geomout}{\file{geom.out}}

%
% Psi Keywords
%
\def\keyword#1{{\tt #1}}

%
% Psi C and Fortran Language elements
%
\def\celem#1{{\tt #1}}
\def\felem#1{{\tt #1}}

%
% Unix stuff
%
\def\unixid#1{{\em #1}} % names of groups and users
\def\shellvar#1{{\tt #1}}

%
% Nice output for function description
%
% Needs 4 arguments: function declaration,
%  description, arguments, and return values
%
% Call \initfuncdesc before using \funcdesc
%
\newcommand{\initfuncdesc}
{\newlength{\lcwidth}
\settowidth{\lcwidth}{Arguments:}
\newlength{\rcwidth}
\setlength{\rcwidth}{\linewidth}
\addtolength{\rcwidth}{-1.0\lcwidth}
\addtolength{\rcwidth}{-6.0\tabcolsep}
}

\newcommand{\funcdesc}[4]{
\celem{#1} \\
#2

\begin{tabular}{lp{\rcwidth}}
Arguments: & #3\\
Returns: & #4
\end{tabular}}


\begin{center}
\ \\
\vspace{2.0in}
{\bf {\Large Installation Manual for the \PSIthree\ Program Package}} \\
\vspace{0.5in}
T.\ Daniel Crawford,$^a$ C.\ David Sherrill,$^b$ and Edward F.\ Valeev$^b$ \\
\ \\
{\em $^a$Department of Chemistry, Virginia Tech, Blacksburg, Virginia 24061-0001} \\
\vspace{0.1in}
{\em $^b$Center for Computational Molecular Science and Technology, \mbox{Georgia 
Institute of Technology,} Atlanta, Georgia 30332-0400} 
\ \\
\vspace{0.3in}
Version of: \today
\end{center}

\thispagestyle{empty}

\newpage
\section{Compilation Prerequisites}

The following external software packages are needed to complile \PSIthree:
\begin{itemize}
\item A well-optimized basic linear algebra subroutine (BLAS) library
  for vital matrix-matrix and matrix-vector multiplication routines.
  We recommend the excellent ATLAS package developed at the University
  of Tennessee.  \htmladdnormallink{{\tt
  math-atlas.sourceforge.net}}{http://math-atlas.sourceforge.net}
\item The linear algebra package (LAPACK), also available from
  netlib.org.  \PSIthree\ makes use of LAPACK's eigenvalue/eigenvector
  and matrix inversion routines.  \htmladdnormallink{{\tt
  www.netlib.org/netlib}}{http://www.netlib.org/netlib}
\item Various GNU utilies: \htmladdnormallink{{\tt
www.gnu.org}}{http://www.gnu.org}
\begin{itemize}
\item {\tt autoconf}
\item {\tt make}
\item {\tt flex}
\item {\tt bison}
\item {\tt fileutils} (esp.\ {\tt install})
\end{itemize}
\item For documentation:
\begin{itemize}
\item {\tt LaTeX}
\item {\tt LaTeX2html} (v0.99.1 or 1.62, including the patch supplied in
psi3/misc)
\end{itemize}
\end{itemize}

\section{Basic Configuration and Installation}

A good directory for the PSI3 source code is /usr/local/src/psi3.
The directory should {\em not} be named {\tt /usr/local/psi}, as that is
the default installation directory unless changed by the {\tt --prefix}
directive (see below).  It should also not have any periods in the path,
e.g., {\tt /usr/local/psi3.2}, because of a bug in {\tt dvips} which will
cause the compilation of documentation to fail.

The following series of steps will configure and build the \PSIthree\
package and install the executables in /usr/local/psi/bin:

\begin{enumerate}
\item {\tt cd \$PSI3} (your top-level \PSIthree\ source directory)
\item {\tt autoconf}
\item {\tt mkdir objdir}
\item {\tt cd objdir}
\item {\tt ../configure}
\item {\tt make}
\item {\tt make tests} (optional, but recommended)
\item {\tt make install}
\item {\tt make doc} (optional)
\end{enumerate}

There is also a perl script, {\tt INSTALL.pl}, in the top-level 
{\tt \$PSI3} source directory which provides an interactive interface 
for installation.

\noindent
You may need to make use of one or more of the following options to
the {\tt configure} script:
\begin{itemize}
\item {\tt --prefix=directory} --- Use this option if you wish to
  install the \PSIthree\ package somewhere other than the default
  directory, {\tt /usr/local/psi}.
\item {\tt --with-cc=compiler} --- Use this option to specify a C
  compiler.  One should use compilers that generate reentrant code,
  if possible. The default search order for compilers is: {\tt gcc},
  {\tt cc}.  (NB: On AIX systems, the search order is {\tt
  cc\_r}, {\tt gcc}.)
\item {\tt --with-cxx=compiler} --- Use this option to specify a C++
  compiler.  One should use compilers that generate reentrant code,
  if possible. The default search order for compilers is: {\tt g++},
  {\tt c++}, {\tt cxx}.  (NB: On AIX systems, the search order is {\tt
  xlC\_r}, {\tt c++}, {\tt g++}.)
\item {\tt --with-fc=compiler} --- Use this option to specify a
  Fortran-77 compiler.  One should use compilers that generate reentrant code,
  if possible. The default search order for compilers is:
  {\tt g77}, {\tt f77}, {\tt fc}, {\tt f2c}.  (NB: On AIX systems, the
  search order is {\tt xlf\_r}, {\tt g77}, {\tt f77}, {\tt
  fc}, {\tt f2c}.)
\item {\tt --with-ar=archiver} --- Use this option to specify an
  archiver.  The default is {\tt ar}.
\item {\tt --with-blas=library} --- Use this option to specify a BLAS
  library.  If your BLAS library has multiple components, enclose the
  file list with single right-quotes, e.g., {\tt
  --with-blas='-lf77blas -latlas'}.
\item {\tt --with-lapack=library} --- Use this option to specify a
  LAPACK library.  If your LAPACK library has multiple components,
  enclose the file list with single right-quotes, e.g., {\tt
  --with-lapack='-llapack -lcblas -latlas'}.
\item {\tt --with-max-am-eri=integer} --- Specifies the maximum
  angular momentum level for the primitive Gaussian basis functions
  when computing electron repulsion integrals.  This is set to
  $g$-type functions (AM=5) by default.
\item {\tt --with-max-am-deriv1=integer} --- Specifies the maximum
  angular momentum level for first derivatives of the primitive
  Gaussian basis functions.  This is set to $f$-type functions (AM=4)
  by default.
\item {\tt --with-max-am-deriv2=integer} --- Specifies the maximum
  angular momentum level for second derivatives of the primitive
  Gaussian basis functions.  This is set to $d$-type functions (AM=3)
  by default.
\item {\tt --with-max-am-r12=integer} --- Specifies the maximum
  angular momentum level for primitive Gaussian basis functions used
  in $r_{12}$ explicitly correlated methods.  This is set to $f$-type
  functions (AM=4) by default.
\item {\tt --with-debug=option} --- This option turns on debugging
  options.  If the argument is omitted, ``{\tt -g}'' will be used by default.
\item {\tt --with-opt=options} --- This option may be used to select
  special optimization flags, overriding defaults.
\item {\tt --with-parallel=sgi} --- This option turns on automatic
  parallelization available with some SGI systems.  Since none of the
  primary developers of \PSIthree\ actually {\em uses} SGI systems at
  present, it seems likely that this option will not work.
\end{itemize}

\section{Detailed Installation Instructions}

This section provides detailed instructions for compiling and
installing the \PSIthree\ package.  

\subsection{Step 1: Configuration}

First, we recommend that you choose for the top-level {\tt \$PSI3} source
directory something other than {\tt /usr/local/psi}; your {\tt \$HOME}
directory or {\tt /usr/local/src/psi3} are convenient choices.  Next,
in the top-level {\tt \$PSI3} source directory you've chosen, first run
{\tt autoconf} to generate the configure script from {\tt configure.in}.
It is best to keep the source code separate from the compilation area,
so you must choose a subdirectory for compilation of the codes.  A simple
option is {\tt \$PSI3/objdir}, which should work for most environments.
However, if you need executables for several architectures, choose more
meaningful subdirectory names.

$\bullet$ The compilation directory will be referred to as {\tt \$objdir}
for the remainder of these instructions.

In {\tt \$objdir}, run the configure script found in the {\tt \$PSI3}
top-level source directory.  This script will scan your system to locate
certain libraries, header files, etc. needed for complete compilation.
The script accepts a number of options, all of which are listed above.
The most important of these is the {\tt --prefix} option, which selects the
installation directory for the executables, the libraries, header files,
basis set data, and other administrative files.  The default {\tt --prefix}
is {\tt /usr/local/psi}.

$\bullet$ The configure script's {\tt --prefix} directory will be referred
to as {\tt \$prefix} for the remainder of these instructions.

\subsection{Step 2: Compilation}

Running {\tt make} (which must be GNU's {\tt 'make'} utility) in {\tt
\$objdir} will compile the \PSIthree\ libraries and executable
modules.

\subsection{Step 3: Testing}

To automatically execute the ever-growing number of test cases after
compilation, simply execute "make tests" in the {\tt \$objdir} directory.
This will run each (relatively small) test case and report the results.
Failure of any of the test cases should be reported to the developers at
psi3@psicode.org.  Note that you must run a "make clean" in 
{\tt \$objdir/tests} to run the test suite again.

After installation (vide infra) you may also run the test suite from the {\tt
\$PSI3/tests} directory using the {\tt driver.test.pl} script.  The results
will be printed to {\tt stdout} as well as the file {\tt test-case-results}.
Note: You {\bf MUST} have the final installation directory, {\tt \$prefix/bin},
in your path in order to run the test cases after installation.  A few
examples for test-case execution are:
\noindent 
\begin{verbatim}
./driver.test.pl             Runs all small and medium tests
./driver.test.pl --ci        Runs all detci tests
./driver.test.pl --cc        Runs all coupled-cluster tests
./driver.test.pl --scf       Runs all hartree-fock tests
./driver.test.pl --sp        Runs all single-point tests
./driver.test.pl --geom      Runs all geometry optimization tests
./driver.test.pl --freq      Runs all harmonic vibrational frequency tests
./driver.test.pl --excite    Runs all excited state tests
./driver.test.pl --small     Runs all low memory, short run time tests
./driver.test.pl --medium    Runs all low memory, med run time tests
./driver.test.pl --large     Runs all high memory, long run time tests
                             (requires considerable disk space, memory, 
                             and time)
./driver.test.pl --all       Runs all tests
./driver.test.pl --clean     Removes all test-case output files.
\end{verbatim} 

You may also run individual test cases using the perl scripts in
the test-case subdirectories.  The \PSIthree\ User's Manual provides
detailed instructions for running calculations. Report bugs to {\tt
psi3@psicode.org}.

\subsection{Step 4: Installation}

Once testing is complete, installation into \$prefix is accomplished by
running {\tt make install} in {\tt \$objdir}.   Executable modules are
installed in {\tt \$prefix/bin}, libraries in {\tt \$prefix/lib}, basis set
data and other control strctures {\tt \$prefix/share}, and documentation
in {\tt \$prefix/doc}.

\subsection{Step 5: Cleaning}

All compilation-area object files and libraries can be removed to save
disk space by running {\tt make clean} in {\tt \$objdir}.

\subsection{Step 6: User Configuration}

After the PSI3 package has been successfullly installed, the user will
need to add the installation directory into their path.  If the package
has been installed in the default location {\tt /usr/local/psi3}, then
in C shell, the user should add something like the following to 
their {\tt .cshrc} file:
\begin{verbatim}
setenv PSI /usr/local/psi3
set path = ($path $PSI/bin)
setenv MANPATH $PSI/doc/man:$MANPATH
\end{verbatim}
The final line will enable the use of the PSI3 man pages.
\begin{verbatim}
\end{verbatim}


\section{Miscellaneous architecture-specific notes}
\begin{itemize}
\item AIX 4.3/5.x in 64-bit environment:
if IBM VisualAge C++ and IBM XL Fortran are used,
one has to manually specify
the {\tt -q64} compiler flag
that enables production of 64-bit executables.
The following configure options should be used:
{\tt --with-cc='cc\_r -q64' --with-cxx='xlC\_r -q64'
 --with-fc='xlf\_r -q64'}. Note that
the reentrant versions of the compilers
are used. Also, we haven't had much luck
using {\tt xlc\_r} because of its
handling of functions with variable argument lists,
use {\tt cc\_r} instead.
\end{itemize}


\end{document}
