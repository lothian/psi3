%
% PSI Programmer's Manual
%
% Essentials of a C Program
%
% David Sherrill, 31 January 1996
% Updates by TDC, 2002.
%

To function as part of the PSI package, a program must incorporate
certain required elements.  This section will discuss the header
files, global variables, and functions required to integrate a C
program into \PSIthree.  Figure \ref{fig:Essential_C_Program} presents
a minimal \PSIthree\ program, whose elements are described below.

\begin{figure}
\begin{verbatim}
                #include <stdio.h>
                #include <libipv1/ip_lib.h>
                #include <libpsio/psio.h>
                #include <libciomr/libciomr.h>

                FILE *infile, *outfile;
                char *psi_file_prefix;

                int main(int argc, char *argv[])
                {
                  extern char *gprgid(void);

                  psi_start(argc-1, argv+1, 0);
                  ip_cwk_add(gprgid());
                  psio_init(); psio_ipv1_config();

                  /* to start timing, tstart(outfile); */
                
                  /* Insert code here */

                  /* to end timing, tstop(outfile); */

                  psio_done();
                  psi_stop();
                }

                char *gprgid(void)
                {
                   /* YOU NEED THE COLON IN THE STRING BELOW */
                   char *prgid = ":CODE_NAME";
                   return(prgid);
                }               
\end{verbatim}
\caption{The essential elements of a \PSIthree\ C-language program.}
\label{fig:Essential_C_Program}
\end{figure}

The required include files are \file{libipv1/ip\_lib.h},
\file{libciomr/libciomr.h}, \file{libpsio/psio.h}, and of course
\file{stdio.h}.  The first of these is for the Input Parser Library,
Version 1 (\file{libipv1.a}), which is described in section
\ref{C_IP}.  The second file contains function prototypes for the C
Math Routines and old-style I/O library, \file{libciomr.a}.  The third
file analogously provides clean interface to functions of the new C
I/O system described in section \ref{C_IO_New}.  The PSI libraries
require that \celem{infile}, \celem{outfile}, and
\celem{psi\_file\_prefix} be global variables.  

The integer function \celem{main()} must be able to handle
command-line arguments required by the \PSIthree\ libraries.  In
particular, all \PSIthree\ modules must be able to pass to the
function \celem{psi\_start()} arguments for the user's input and
output filenames, as well as a global file prefix to be used for
naming standard binary and text data files.  (NB: the default names
for user input and output are \inputdat\ and \outputdat, respectively,
though any name may be used.) The current standard for command-line
arguments is for all module-specific arguments ({\em e.g.},
\celem{--quiet}, used in \module{detci}) {\em before} the input,
output, and prefix values.  The \celem{psi\_start()} function expects
to find {\em only} these last three arguments at most, so the
programmer should pass as \celem{argv[]} the pointer to the first
non-module-specific argument.  The above example is appropriate for a
\PSIthree\ module that requires no command-line arguments apart from
the input/output/prefix globals.  See the \PSIthree\ modules
\module{input} and \module{detci} for more sophisticated examples.
The final argument to \celem{psi\_start()} is an integer whose value
indicates whether the output file should be overwitten (1) or appended
(0).  Most \PSIthree\ modules should choose to append.

The \celem{psi\_start()} function initializes the user's input and
output files and sets the global variables \celem{infile},
\celem{outfile}, and \celem{psi\_file\_prefix}, based on (in order of
priority) the above command-line arguments or the environmental
variables \celem{PSI\_INPUT}, \celem{PSI\_OUTPUT}, and
\celem{PSI\_PREFIX}.  The value of the global file prefix can also be
specified in the user's input file.  The \celem{psi\_start()} function
will also initialize the input parser and sets up a default keyword
tree (described in detail in section \ref{C_IP}).  This step is
required even if the program will not do any input parsing, because
some of the functionality of the input parser is assumed by
\library{libciomr.a} and \library{libpsio.a}.  For instance, opening a
binary file via \celem{psio\_open()} (see section \ref{C_IO_New})
requires parsing the \keyword{files} section of the user's input so
that a unit number (e.g.~52) can be translated into a filename.

The \celem{psi\_stop()} function shuts down the input parser and closes
the user's input and output files.

Timing information (when the program starts and stops, and how much
user, system, and wall-clock time it requires) can be printed to the
output file by adding calls to \celem{tstart()} and \celem{tstop()}
(from \library{libciomr.a}).

The sole purpose of the simple function \celem{gprgid()} is to provide
the input parser a means to determine the name of the current program.
This allows the input parser to add the name of the program to the
input parsing keyword tree.  This function is used by
\library{libpsio.a}, though the functionality it provides is rarely
used.

NB: The library \library{libciomr.a} contains older I/O functions that
have been superceded by functions in \library{libpsio.a}.  However,
you are encouraged to use the many non-I/O functions in
\library{libciomr.a}.
