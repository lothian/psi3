%
% History of Psi
%
% Daniel Crawford, 24 January, 1996
%

The PSI suite of {\em ab initio} quantum chemistry programs is the result
of an ongoing attempt by a cadre of graduate students, postdoctoral
associates, and professors to produce code that is efficient but also
easy to extend to new theoretical methods.  Significant effort has been
devoted to the development of libraries which are robust and easy to use.
Some of the earliest contributions to what is now referred to as ``PSI''
include a direct configuration interaction (CI) program (Robert Lucchese,
1976, now at Texas A\&M), the well-known graphical unitary group CI program
(Bernie Brooks, 1977-78, now at N.I.H.), and the original integrals code
(Russ Pitzer, 1978, now at Ohio State).  From 1978-1987, the package was
know as the {\tt BERKELEY} suite, and after the Schaefer group moved to the
Center for Computational Quantum Chemistry at the University of Georgia,
the package was renamed {\tt PSI}.  Thanks primarily to the efforts of Curt
Janssen (Sandia Labs, Livermore) and Ed Seidl (LLNL), the package was
ported to UNIX systems, and substantially improved with new input formats
and a C-based I/O system.

Beginning in 1999, an extensive effort was begun to develop \PSIthree\
--- a {\tt PSI} suite with a completely new face.  As a result of this
effort, all of the legacy Fortran code was removed, and everything was
rewritten in C and C++, including new integral/derivative integral,
coupled cluster, and CI codes.  In addition, new I/O libraries have
been added, as well as an improved checkpoint file structure and greater
automation of typical tasks such as geometry optimization and frequency
analysis.  The package has the capability to determine wavefunctions,
energies, analytic gradients, and various molecular properties based on
a variety of theories, including spin-restricted, spin-unrestricted, and
restricted open-shell Hartree-Fock (RHF, UHF, and ROHF); configuration
interaction (CI) (including a variety of multireference CI's and full
CI); coupled-cluster (CC) including CC with variationaly optimized
orbitals; second-order M{\o}ller-Plesset perturbation theory (MPPT)
including explicitly correlated second-order M{\o}ller-Plesset energy
(MP2-R12); and complete-active-space self-consistent field (CASSCF)
theory.  By January 2008, all of the C code in \PSIthree\ was 
converted to C++ to enable a path toward more object-oriented design
and a single-excecutable framework that will facilitate code reuse and 
ease efforts at parallelization.  At this same time, all of the legacy I/O
routines from {\tt PSI2} were removed, greatly streamlining the
\library{libciomr.a} library.

