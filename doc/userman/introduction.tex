\section{Introduction} \label{introduction}

\subsection{Overview} 

This manual explains how to use the \PSIthree\ suite of {\em ab initio}
quantum chemical programs.  In this section, we provide an overview of
some of the features of \PSIthree\ along with the prerequisite steps for
running calculations.  Section \ref{tutorial} provides a brief tutorial to
help new users get started.  Section \ref{input} offers further details
into the structure of \PSIthree\ input files and a discussion of some of
the most important options.  Later sections deal with the different types
of computations which can be done using \PSIthree\ (e.g., Hartree-Fock,
MP2, coupled-cluster) and general procedures such as geometry optimization
and vibrational frequency analysis.  The appendix will eventually include a
description of the input keywords and command-line options for each module,
as well as numerous examples of \PSIthree\ input and basis set files.
For the latest \PSIthree\ documentation, check \htmladdnormallink{{\tt
www.psicode.org}} {http://www.psicode.org/}.

The \PSIthree\ package was developed to perform high-accuracy quantum
mechanical computations on challenging chemical species and to provide an
infrastructure for the development of new theoretical techniques.  Hence,
it has a very flexible input scheme which allows non-standard computations,
and it is easily adapted to enable new capabilities.

The following citation should be used in any publication utilizing the
\PSIthree\ program package:

\begin{quotation}
\noindent
T. Daniel Crawford, C. David Sherrill, Edward F. Valeev, Justin
T. Fermann, Rollin A. King, Matthew L. Leininger, Shawn T. Brown,
Curtis L. Janssen, Edward T. Seidl, Joseph P. Kenny, and Wesley D. Allen,
{\em J. Comput. Chem.}, in press.

\end{quotation}

\subsection{Obtaining and Installing \PSIthree}
\label{installation}

The latest version of the \PSIthree\ program package may be obtained at
\htmladdnormallink{{\tt www.psicode.org}}{http://www.psicode.org}.  The
source code is available as a gzipped tar archive (named, for example, {\tt
psi3.X.tar.gz}), and binaries may be available for certain architectures.
For detailed installation and testing instructions, please refer to the
the \PSIthree\ Installation Manual, available as part of the package or
at the \PSIthree\ website above.

\subsection{Supported Architectures}
The majority of \PSIthree\ was developed on IBM RS/6000/AIX
and x86/GNU Linux workstations. The complete list of
tested architectures to which \PSIthree\ has
been ported is shown in Table \ref{table:ports}.
\begin{table}[h]
\caption{Platforms on which \PSIthree\ has been installed successfully.}
\label{table:ports}
\begin{center}
\begin{tabular}{ll} \hline\hline
Architecture              &  Notes \\ \hline
Compaq Alpha Tru64 UNIX   & 64-bit mode \\
IBM AIX 4.3.3, 5.x on PowerPC & 64-bit mode \\
Linux on Intel/AMD x86, AMD x86-64  & 32 and 64-bit\\
Apple OS X (Darwin) on PowerPC & \\
SGI IRIX64 ($>$6.5.15)    & 64-bit \\ \hline\hline
\end{tabular}
\end{center}
\end{table}
If you don't find your system in the Table, there's a good chance
that you will be able to install \PSIthree\ on your system
if you have the prerequisite tools and math and utility libraries described 
in the installation manual.

\subsection{Capabilities}
\PSIthree\ can perform {\em ab initio} computations employing
basis sets of up to 32768 contracted Gaussian-type functions of
virtually arbitrary orbital quantum number.
\PSIthree\ can recognize and exploit the largest Abelian subgroup of the
point group describing the full symmetry of the molecule.
Table \ref{table:methods} displays the range of theoretical
methods available in \PSIthree .

\begin{table}
\caption{Summary of theoretical methods available in \PSIthree.} \label{table:methods}
\parsep 10pt
\begin{center}
\begin{tabular}{lccc} \hline\hline
Method                & Energy & Gradient & Hessian \\ \hline
RHF SCF               & Y & Y & Y \\
ROHF SCF              & Y & Y & N \\
UHF SCF               & Y & N & N \\
HF DBOC               & Y & N & N \\
CIS/RPA/TDHF          & Y & N & N \\
TCSCF                 & Y & Y & N \\
CASSCF                & Y & Y & N \\
RASSCF                & Y & Y & N \\
RAS-CI                & Y & N & N \\
RAS-CI DBOC           & Y & N & N \\
RHF MP2               & Y & Y & N \\
UHF/ROHF MP2          & Y & N & N \\
RHF MP2-R12           & Y & N & N \\
RHF/UHF/ROHF CCSD     & Y & Y & N \\
RHF/UHF/ROHF CCSD(T)  & Y & N & N \\
RHF/UHF/ROHF EOM-CCSD & Y & Y & N \\
\hline\hline
\end{tabular}
\end{center}
\end{table}
Geometry optimization (currently restricted to true minima on the potential
energy surface) can be performed using either analytic gradients
or energy points.  Likewise, vibrational frequencies can be 
computed using analytic second derivatives, by finite
differences of analytic gradients, or finite differences of energies.
\PSIthree\ can also compute an extensive list of one-electron properties.

\subsection{Technical Support} The \PSIthree\ package is
distributed for free and without any guarantee of reliability,
accuracy, suitability for any particular purpose.  No obligation
to provide technical support is expressed or implied.  As time
allows, the developers will attempt to answer inquiries directed to
\htmladdnormallink{{\tt crawdad@vt.edu}}{mailto:crawdad@vt.edu}.
For bug reports, specific and detailed information, with example
inputs, would be appreciated.  Questions or comments regarding
this user's manual may be sent to \htmladdnormallink{{\tt
sherrill@gatech.edu}}{mailto:sherrill@gatech.edu}.



