\subsection{Capabilities}
\PSIthree\ can perform {\em ab initio} computations employing
basis sets of up to 32768 contracted Gaussian-type functions of
virtually arbitrary orbital quantum number.
\PSIthree\ can recognize and exploit the largest Abelian subgroup of the
point group describing the full symmetry of the molecule.
Table \ref{table:methods} displays the range of theoretical
methods available in \PSIthree .
\begin{table}[h]
\caption{Summary of theoretical methods available in \PSIthree.} 
\label{table:methods}
\parsep 10pt
\begin{center}
\begin{tabular}{lccc} \hline\hline
Method      & Energy & Gradient & Hessian \\ \hline
RHF SCF     & Y & Y & Y \\
ROHF SCF    & Y & Y & N \\
UHF SCF     & Y & N & N \\
CIS/RPA/TDHF& Y & N & N \\
TCSCF       & Y & Y & N \\
CASSCF      & Y & Y & N \\
RAS-CI      & Y & N & N \\
RHF-MP2     & Y & N & N \\
RHF-MP2-R12 & Y & N & N \\
RHF-CCSD    & Y & Y & N \\
ROHF-CCSD   & Y & Y & N \\
VOO-CCD     & Y & Y & N \\ 
RHF-CCSD(T) & Y & N & N \\ 
UHF-CCSD(T) & Y & N & N \\ 
RHF-EOM-CCSD& Y & N & N \\ 
ROHF-EOM-CCSD& Y & N & N \\ 
\hline\hline
\end{tabular}
\end{center}
\end{table}
Geometry optimization (currently restricted to true minima on the potential
energy surface) and vibrational frequency computations can be performed
with the methods for which analytic gradients and Hessian, respectively, are
available.  Methods for which analytic gradients are not yet implemented
may still be used for geometry optimization if the gradients are computed
numerically from energies; this is automated by the optking program.
Numerical computation of Hessians is being developed.
\PSIthree\ can also compute an extensive list of one-electron properties.
Finally, it should be mentioned that whenever \PSIthree\ is used,
it should be cited fully (see appendix).

\subsection{Before You Obtain \PSIthree: Prerequisites}
The majority of \PSIthree\ was developed on IBM RS/6000//AIX
and Intel i386/Linux workstations. The complete list of
tested processor/OS combinations to which \PSIthree\ has
been ported is shown in Table \ref{table:ports}.
\begin{table}
\caption{Platforms on which \PSIthree\ has been installed successfully.} \label{table:ports}
\begin{tabular}{lll} \hline\hline
Hardware & Operating System(s) & Special Notes \\ \hline
Compaq Alpha & Compaq TrueUNIX64 & 64-bit mode \\
IBM RS/6000 & AIX 3.2.5, AIX 4.1-4.3.0 &
IBM C, C++, and Fortran compilers\\
IBM PowerPC & AIX 4.3.2 & 
IBM C, C++, and Fortran compilers;
64-bit mode\\
Intel ix86 & Linux 2.2 & \\
SGI Origin 2000 & IRIX64 6.5 & 64-bit \\ \hline\hline
\end{tabular}
\end{table}
If you don't find your system in the Table, there's a good chance
that you will be able to install \PSIthree\ on your system
if you have all of the following:

\subsection{Before You Run \PSIthree}
Every user needs to configure her or his
shell environment prior to running \PSIthree:
\begin{enumerate}
\item add the location of the binaries to {\tt \$PATH}
\item add the location of the manpages to {\tt \$MANPATH}
\end{enumerate}
In C-shell it may be achieved like this:
\begin{verbatim}
set psipath = /put/psi/installation/directory/here
setenv PATH $psipath`$psipath/bin/host.sh`/bin:$PATH
setenv MANPATH $psipath/doc/man:$MANPATH
\end{verbatim}
Note that in previous versions of PSI, the installation directory
was frequently in a subdirectory of the source distribution directory.
This is no longer the case.  Now, \file{psipath} is commonly something
like \file{/usr/local/psi3} while the source directory will be elsewhere.

