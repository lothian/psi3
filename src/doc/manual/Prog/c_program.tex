%
% PSI Programmer's Manual
%
% Essentials of a C Program
%
% David Sherrill, 31 January 1996
%

To function as part of the PSI package, a program must incorporate
certain required elements.  This section will discuss the header
files, global variables, and functions 
required to integrate a C program into \PSIthree; analogous elements are required
for Fortran programs.  Figure \ref{fig:Essential_C_Program}
presents the minimal \PSIthree\ program written in C.

\begin{figure}
\begin{verbatim}
                #include <stdio.h>
                #include <ip_libv1.h>
                #include <psio.h>
                #include <libciomr.h>


                void main()
                {
                  FILE *infile, *outfile;

                  ffile(&infile,"input.dat",2);
                  ffile(&outfile,"output.dat",1);
                  ip_set_uppercase(1);
                  ip_initialize(infile,outfile);
                  ip_cwk_add(":DEFAULT");
                  ip_cwk_add(prgid);
                  psio_init();

                  /* to start timing, tstart(outfile) */
                
                  /* Insert code here */

                  /* to end timing, tstop(outfile) */

                  psio_done();
                  fclose(infile);
                  fclose(outfile);
                  ip_done();
                }

                char *gprgid()
                {
                   char *prgid = "CODE_NAME";
                   return(prgid);
                }               
\end{verbatim}
\caption{The essential elements of a \PSIthree\ C-language program.}
\label{fig:Essential_C_Program}
\end{figure}

The required include files are \file{ip\_libv1.h}, \file{libciomr.h}, \file{psio.h},
and of course \file{stdio.h}.  The first of these is for the Input Parser Library,
Version 1 (\file{libipv1.a}), which is described in section \ref{C_IP}.  The
second file contains function prototypes for the C Math Routines and old-style I/O
library, \file{libciomr.a}, described in section \ref{C_IO}.  The header
\file{libciomr.h} was added to \PSItwo\ sometime in 1994, and older C programs
may not include this file; however, it is recommended for all new programs
in order to provide type checking for functions and function arguments.
The third file analogously provides clean interface to functions of the new C I/O system
described in section \ref{C_IO_New}.
Formerly, PSI programs required the file pointers \celem{infile} and \celem{outfile}
to be global variables.  However, it seems that this is no longer necessary.

The functions in \celem{main()} are required to set up and shut down the
input parser, and they have already been described in detail in section
\ref{C_IP}.  Even if the program will not do any input parsing, it may be
necessary to include these statements, because some of the functionality of
the input parser is assumed by \library{libciomr.a} and \library{libpsio.a}.
For instance, opening a binary file via \celem{psio\_open()} requires
parsing the \keyword{files} section of
\inputdat\ so that a unit number (e.g.~52) can be translated into a
filename (e.g. \file{h2o.52}).  Timing information (when the program starts 
and stops, and how much user, system,
and total time it requires) can be printed to the output file by adding
calls to \celem{tstart()} and \celem{tstop()} as indicated in the figure.

The sole purpose of the simple function \celem{gprgid()} is to provide the
input parser a means to determine the name of the current program.  This is
necessary so that the input parser can add the name of the program to the
input parsing keyword tree.  This function seems to be used in
the \celem{get\_file\_info()} function of \library{libciomr.a} and a few functions of
\library{libpsio.a}, which allow for
a \keyword{files} section within a section of input dedicated to a specific
program, such as \PSIcscf.  This feature of \PSIthree\ does not seem to be
commonly used.

{\em One note on choice of the I/O system to use. It is mandatory that
every new \PSIthree\ program uses only the new binary I/O system described in
section \ref{C_IO_New}. However, you are encouraged to use some non-binary
I/O functions in \library{libciomr.a}, like \celem{ffile()}, which provide
a solid interface to standard C I/O functions.}
