%
% PSI Programmer's Manual
%
% CVS Revision Control Section
%
% Daniel Crawford, 1 February, 1996
%
There are two ways to start developing for \PSIthree:
\begin{itemize}
\item obtain a tarball containing \PSIthree\ distribution, run installation script,
and then just add your source code to the resulting local \PSIthree\ source tree
or modify existing \PSIthree\ modules. Pertinent installation steps are described
in \PSIthree\ User's Manual.
\item use the CVS version control system to ``{\em check out}'' a copy of \PSIthree\ 
source from the main repository, add your new code to the local source tree or modify
existing \PSIthree\ modules, and then make changes and additions available to
other parties by checking them back into the main repository;
\end{itemize}
The second method is preferred if you are to develop seriously for \PSIthree.
Note that, although legally you do not have to, we encourage
you to make results of your hard work available to everyone. If the decision
to move PSI under GNU Public Liscense is made then you will be {\em required}
to do so.

This section describes how to access and modify \PSIthree\ code via
the CVS version control system. For further information,
please see the cvs(1) info page.

\subsection{Introduction to \PSIthree\ repository}
The main repository for the \PSIthree\ Source code is
\file{/home/xerxes/psi3/master}.  Under this root, all of the source code is
maintained under CVS control.  In order to conveniently examine/alter the
source code for any part of \PSIthree, one must first ``check out'' a copy of the
code.  This personal copy may be altered, or code may be added to it.
After all of the user's changes have been tested thoroughly, this new
version of the source code may be ``checked in'' to the main respository,
and then later installed into the local code tree (\file{/home/xerxes/psi3}).
Any user who is a member of the \unixid{psiadm} group may check out/in code
from/to to main repository.  Any user with access to the \unixid{psi} user may
compile/install code under the CCQC-wide installation tree.  CVS allows users to check
out any stored version of the code.  For example, if a user wishes to
obtain a copy of the code as it stood on a given date, or even at a given
time, CVS provides commands for accomplishing this task.

\subsection{Prerequisite steps to accessing the master respository}
Only users who are members of the \unixid{psiadm} group may check out copies of the
Psi source code.

The following steps will check out all of the essential Psi code needed for basic,
independent compilation and testing:
\begin{enumerate}
\item Make sure the environmental variable \shellvar{CVSROOT} is set to
\file{/home/xerxes/psi3/master}. For example, for csh
\begin{verbatim}
setenv CVSROOT /home/xerxes/psi3/master
\end{verbatim}
will set this correctly. This line should be placed in the user's shell
resource file. For users at remote sites, the form should be 
\begin{verbatim}
USER@CCQC_MACHINE_NAME:/home/xerxes/psi3/master 
\end{verbatim}
If you do not what {\tt CCQC\_MACHINE\_NAME} should be set to --
contact \PSIthree\ developers at \htmladdnormallink{{\tt mailto:psi@ccqc.uga.edu}}
{mailto:psi@ccqc.uga.edu}
\item In the desired location for the root of the local
\PSIthree\ source tree,
\begin{verbatim}
cvs co psi3
\end{verbatim}
\end{enumerate}
These two steps create a complete source code tree under \file{psi3}
with all necessary content. Next, you might simply edit \file{PSI3Install} and
run it as described in \PSIthree\ User's Manual, but let us instead compile
and install \PSIthree\ manually. It is highly recommended that you keep
the local source tree separate from the compilation tree, i.e. rename 
\file{psi3} to something like \file{psi3\_dist}. This will allow you to
wipe out products of compilation and installation processes at any point
and start all over again. We will assume that user has a local
source tree checked out to \shellvar{\$HOME}\file{/psi3\_dist}.

First, it is necessary to run the autoconf program, which will take the
information in \file{psi3\_dist/configure.in} and generate a shell
script named \file{configure} which will later be used to determine
a large amount of architecture-dependent information needed for correct
compilation. To do this, in the directory \shellvar{\$HOME}\file{/psi3\_dist} run
\begin{verbatim}
autoconf 
\end{verbatim}
Notice that files named \file{Makefile.in}, \file{MakeVars.in}, etc.
exist after the check out sequence above. These files are input for
the newly-created \file{configure} script which will generate Makefiles from them.

\subsection{Initial compilation of utilities, libraries, and binaries}
Before one may compile any of the \PSIthree\ binaries (e.g. \PSIcscf, \PSIcints, etc.)
it is necessary to specify the installation directory for \PSIthree.
The installation directory may be located anywhere. This offers a lot of
flexibility to the programmer. Let's say you maintain \PSIthree\ for
a group of users and you also contribute to \PSIthree. Let's say you do most of your
development on one machine, and want to make up-to-date \PSIthree\ codes
available to the group as well. An obvious but not the most efficient way to
maintain \PSIthree\ in such
an environment would be to check out two local copies of \PSIthree\ source,
one on your development machine and one on a public machine.
However, experience shows that having many checked out copies of \PSIthree\ may
increase time spent on code maintenance and decrease time spent on code
development and documentation.
A more elegant way of doing this that would be to keep only one source directory
(on your development machine), but have two installation directories:
one accessible to the group, and one accessible to you only. Then you could make
any changes to your own binaries without sacrificing stability of the \PSIthree\ code used
by the group. Each installation directory may contain one or more architecture-specific
compilation directories. Hence if your group uses a variety of computer architectures,
you will be able to keep \PSIthree\ binaries for all of them in one location.

Note that throughout this document, \file{make} automatically implies \file{gmake} (the
GNU Project's make utility).

The following steps will create an installation directory
\shellvar{\$HOME}\file{/psi3}, under the installation directory
create an architecture-specific directory
whose name is determined by the \file{host.sh} script,
run the \file{configure} script to determine locations and/or existence of
essential libraries and utilities and generate the appropriate make and other files,
and compile all of the essential utilities, libraries and binaries
(we will assume C-shell here):
\begin{enumerate}
\item {\tt cd \$HOME/psi3}
\item {\tt setenv ARCH `\$HOME/psi3\_dist/host.sh`}
\item {\tt mkdir \$ARCH}
\item {\tt cd \$ARCH}
\item {\tt \$HOME/psi3\_dist/configure --prefix=\$HOME/psi3 --verbose}
\item {\tt make install\_host}
\item {\tt cd include}
\item {\tt make install}
\item {\tt cd ../lib}
\item {\tt make install}
\item {\tt cd ../src/util}
\item {\tt make install}
\item {\tt cd ../lib}
\item {\tt make install}
\item {\tt cd ../bin}
\item {\tt make install}
\item {\tt cd ../doc}
\item {\tt make install}
\end{enumerate}
The last two steps of this procedure also compile and install
The \PSIthree\ User's and Programmer's manuals.

After running \file{configure}, and prior to compilation of the libraries and binaries, the
programmer may wish to configure some parts of \PSIthree\ manually. In order
to allow the \PSIthree\ I/O routines to read and write beyond the default
file size limit of 4.0 GB, the programmer needs to edit files
\shellvar{\$HOME}\file{/psi3/}\shellvar{\$ARCH}\file{/src/lib/io/param.h} and
\shellvar{\$HOME}\file{/psi3/}\shellvar{\$ARCH}\file{/src/lib/libciomr/iomrparam.h}.
In order to change the angular momentum limit from the default of $g$,
one needs to edit \shellvar{\$HOME}\file{/psi3/}\shellvar{\$ARCH}\file{/src/lib/libint/input.dat}.
The integrals code \PSIcints\ may also require some tuning (edit
\shellvar{\$HOME}\file{/psi3/}\shellvar{\$ARCH}\file{/src/bin/cints/defines.h}).
However, it is strongly recommended that this be done only by \PSIthree\ experts, as such changes
may introduce errors. If you make this modification, you will need to re-edit
these files to make these changes again if you run config.status or configure
in the architecture-dependent directory.

The newly-compiled utilities, libraries, binaries, and manuals
are installed under \shellvar{\$HOME}\file{/psi3}
(this is why the {\tt --prefix} option to \file{configure} is used) in the architecture-specific
subdirectory. Each time you wish to compile any part of the code for a machine
of a different architecture, you must follow the above sequence of steps in order
to keep object files, binaries, and libraries separated appropriately. After these
steps, the \shellvar{\$HOME}\file{/psi3} directory should contain the following: 
\begin{verbatim}
bin  doc  i586-pc-linux-gnu  include  share
\end{verbatim}
though the name of the architecture-specific directory may be different. See
subsection \ref{psitree} for a description of what may be found
in each of these directories. 

\subsection{Checking in altered \PSIthree\ binaries or libraries}
Only members of the \unixid{psiadm} group may check altered code into the main
repository.

If the programmer has made changes to Psi binaries or libraries which already
exist, one of two series of steps is necessary to check these changes in to the
main repository. The first series may be followed if all changes have been made
only to files which already exist in the current version. The second series should
be followed if new files must be added to the code in the repository.
\begin{itemize}
\item No new files need to be added to the repository. We will use
\library{libciomr} as an example. 
\begin{enumerate}
\item {\tt cd \$HOME/psi3\_dist/src/lib/libciomr}
\item {\tt cvs ci}
\item Edit the comment file that CVS provides. 
\end{enumerate}
\item New files must be added to the repository. Again, we use \library{libciomr}
as an example. Suppose the new file is named \file{great\_code.c} .
\begin{enumerate}
\item {\tt cd \$HOME/psi3\_dist/src/lib/libciomr} 
\item {\tt cvs add great\_code.c} 
\item {\tt cvs ci}
\item Edit the comment file that CVS provides.
\end{enumerate}
\end{itemize}

The \file{cvs ci} command in both of these sequences will examine all of the code in
the current \file{libciomr} directory against the current version of the code in the
main repository. Any files which have been altered (and for which no conflicts
with newer versions exist!) will be identified and checked in to the main
repository, as well as the new file in the second series. CVS will open a
comment file for editing; you should enter a description of the changes you have
made to the code here, as well as in the code itself. Exit the editing of the
comment file, and CVS will check the new code into the main repository.

After these steps have been followed, the programmer should make changes available
to the group. At CCQC it means convincing a user
with access to the \unixid{psi} user to install the code
into the CCQC installation tree currently located under
\file{/home/xerxes/psi3}.

\subsection{Adding entirely new code to the main \PSIthree\ repository} \label{checkin_new}
If the programmer is adding a new binary to the \PSIthree\ repository, a number of
important conventions should be followed:
\begin{enumerate}
\item The directory containing the new code should be given a name which
matches the name of the installed code (e.g. if the code will be installed
as \module{newcode}, the directory containing the code should be named
\file{newcode}). New binaries must be constructed in
\shellvar{\$HOME}\file{/psi3\_dist/src/bin} and libraries in
\shellvar{\$HOME}\file{/psi3\_dist/src/lib} of the user's local source tree. 
\item The Makefile should be converted to an input file for the configure
script (see any of the current \PSIthree\ binaries for an example) and should
follow the conventions set up in all of the current \PSIthree\ 
\file{Makefile.in}'s. This
includes use of \file{MakeVars} and \file{MakeRules} files.
\item New binaries should be added to the list contained in
\shellvar{\$HOME}\file{/psi3\_dist/src/bin/Makefile.in} so that they
will be compiled automatically when a full compilation of
the \PSIthree\ distribution occurs. This step is included in the sequence below. 
\item A documentation page should be included with the new code (see
section \ref{Documentation} for more information). As a general rule,
if the code is not ready to have a documentation page, it is not ready
to be installed in \PSIthree. 
\item The \file{configure.in} file must be altered so that
users may check out copies of the new code and so that the \file{configure}
script will know to create the Makefile for the new code. These steps
are included in the sequence below. 
\item Fortran source files should be named with a \file{.F} suffix so that they
will be processed by \file{psipp}.
\end{enumerate}

Assume the new code is a binary and is named \module{great\_code}. The directory
containing the new code must contain only those files which are to be checked
in to the repository! Then the following steps will check in a new
piece of code to the main repository:
\begin{enumerate}
\item {\tt cd \$HOME/psi3\_dist/src/bin}
\item {\tt cvs add great\_code}
\item Answer ``y'' when CVS asks if you wish to create the new directory in
the repository. 
\item {\tt cd great\_code}
\item {\tt cvs add *}
\item {\tt cvs ci}
\item Edit the comments file that CVS provides. 
\item {\tt cd \$HOME/psi3\_dist}
\item Edit the \file{configure.in} script and add \file{great\_code} to the list. 
\item {\tt cvs ci}
\item Edit the comments file that CVS provides. 
\item {\tt autoconf} 
\item {\tt cd \$HOME/psi3\_dist/src/bin} 
\item Edit \file{Makefile.in} and add \file{great\_code} to the list. 
\item {\tt cvs ci}
\item Edit the comments file that CVS provides. 
\end{enumerate}
At this point, all of the code has been properly checked in. However, you must
test to make sure that the code can be checked out by other programmers, and
that it will compile correctly after being checked out. The following steps will
store your personal version of the code, check out the new code, and
test-compile it:
\begin{enumerate}
\item {\tt cd \$HOME/psi3\_dist/src/bin}
\item {\tt mv great\_code great\_code.bak}
\item {\tt cd \$HOME}
\item {\tt cvs co psi3\_dist/src/bin/great\_code}
\item {\tt cd psi3/`psi3/bin/host.sh`}
\item {\tt \$HOME/psi3\_dist/configure --prefix=\$HOME/psi3 --verbose}
\item {\tt cd src/bin/great\_code}
\item {\tt make install}
\end{enumerate}
Your original version of the code remains under \file{great\_code.bak}, but should be
no longer necessary. Note that it is necessary to re-run \file{configure} explicitly,
instead of just running \file{config.status}, because the \file{config.status} file
contains no information about the new code.

\subsection{Updating checked out code}
If the code in the main repository has been altered, users' checked out copies of
the code will of course not automatically be updated to relect the new changes.
Users must do this themselves to keep their versions up-to-date.

In general, it is only necessary to execute the following steps in order to
completely update your personal checked out version of the \PSIthree\ code:
\begin{enumerate}
\item {\tt cd \$HOME/psi3\_dist}
\item {\tt cvs update}
\end{enumerate}
This will examine each entry in your checked out code hierarchy and compare it
to the most recent version in the main repository. When that in the main
repository is more recent, your version of the code is updated. If you have made
changes to your version, but the version in the main repository has not changed,
the altered code will be identified to you with an ``M''. If you have made
changes to your version of the code, and one or more newer versions have been
updated in the main repository, CVS will examine the two versions and attempt
to merge them -- this process usually has conflicts however, and is rarely
successful. You will be notified of any conflicts that arise and it will be up to
you to resolve them.

If entirely new code has been added to the main repository, the above steps will
not automatically check out the new code. To do this, the following steps should
be executed:

Assume the name of the new code is \file{great\_code}. 
\begin{enumerate}
\item {\tt cd \$HOME/psi3\_dist}
\item {\tt cvs update -d}
\item {\tt autoconf}
\item {\tt cd \$HOME/psi3/`\$HOME/psi3/bin/host.sh`} 
\item {\tt \$HOME/psi3\_dist/configure --prefix=\$HOME/psi3 --verbose}
\item {\tt cd src/bin/great\_code}
\item {\tt make install}
\end{enumerate}
This will check out any new code that has been added to the repository since the last time the
\file{cvs update} command was issued, correctly update the \file{configure} script
(assuming the programmer who added the code followed the steps from section \ref{checkin_new}
and edited the \file{configure.in} file), re-configure for the new code (note that
you must run \file{configure} and not \file{config.status} here), and then compile the
code.

\subsection{Removing code from the repository}
If alterations of libraries or binaries under Psi involves the deletion of source
code files from the code, these must be explicitly removed through CVS.

The following steps will remove a source code file named \file{bad\_code.F} from a
binary named \module{great\_code}:
\begin{enumerate}
\item {\tt cd \$HOME/psi3\_dist/src/bin/great\_code}
\item {\tt rm bad\_code.F}
\item {\tt cvs remove bad\_code.F}
\item {\tt cvs ci}
\end{enumerate}
Then edit the comments file that CVS provides. 

\subsection{Checking out older versions of the code}
It is sometimes necessary to check out older versions of a piece of code.
Assume we wish to check out an old version of \PSIinput. If this
is the case, the following steps will do this:
\begin{enumerate}
\item {\tt cd \$HOME/psi3\_dist/src/bin/input}
\item {\tt cvs update -D"2 months ago"}
\end{enumerate}
This will check the main repository and provide you with the code as it stood 2
months ago. CVS is quite impressive in this respect. It will accept all sorts of
input to the -D option. You could even use \file{-D"a fortnight ago"} and CVS
would get the correct version. (But it doesn't get \file{"two fortnights ago"} right.
:) You can always get the recent version back by simply using \file{-D"now"}. Note
that subsequent updates of the current code will use the same date you give
with the \file{-D} option. 

\subsection{The structure of the \PSIthree\ repository, source, and installation trees} \label{psitree}
As mentioned above, the main repository tree may be found in
\file{/home/xerxes/psi3/master/psi3}, which we will call \shellvar{\$ROOT},
for convenience. \shellvar{\$ROOT} contains a number of important
installation scripts, and configure input files.
There are three pertinent subdirectories of \shellvar{\$ROOT}:
\begin{itemize}
\item \shellvar{\$ROOT}\file{/lib} -- contains the CVS-stored non-architecture-dependent ``library'' data. This
currently includes the psi man page nroff macro file, \file{macro.psi}, the basis
set data, \file{pbasis.dat}, the program execution order, \file{psi.dat}, and tmp disk
information for the local machines (CCQC by default), \file{tmpdisk.dat}. 
\item \shellvar{\$ROOT}\file{/include} -- contains the CVS-stored architecture-independent header files.
\item \shellvar{\$ROOT}\file{/src} -- contains the CVS-stored source code for binaries,
libraries, and utilities. \\
\shellvar{\$ROOT}\file{/src} has the following subdirectories:
\begin{itemize}
\item \shellvar{\$ROOT}\file{/src/util} -- contiains the CVS-stored source code for \module{psipp}, \module{tmpl}.
\item \shellvar{\$ROOT}\file{/src/lib} -- contains the CVS-stored source code for all of the \PSIthree\ libraries,
including \library{libpsio}, \library{libipv1}, etc.
\item \shellvar{\$ROOT}\file{/src/bin} -- contains the CVS-stored source code for all of the \PSIthree\ 
binaries. 
\end{itemize}
\end{itemize}

An example of an installation tree may be found in \file{/home/xerxes/psi3}, which we will call
\shellvar{\$LOCAL}, for convenience.
\shellvar{\$LOCAL} has the following pertinent subdirectories:
\begin{itemize}
\item \shellvar{\$LOCAL}\file{/bin} -- contains non-architecture-dependent executables and
shell scripts. Currently this consists of \file{host.sh}, \file{config.guess}, and \file{config.local}, which
together define a canonical name for the system.
\item \shellvar{\$LOCAL}\file{/doc} -- contains \PSIthree\ documentation files.
\item \shellvar{\$LOCAL}\file{/include} -- contains non-architecture-dependent header files.
\item \shellvar{\$LOCAL}\file{/share} -- contains non-architecture-dependent ``library'' data.
This currently includes the psi man page nroff macro file, \file{macro.psi}, the basis
set data, \file{pbasis.dat}, the program execution order, \file{psi.dat}, and tmp disk
information for the local machines (CCQC by default), \file{tmpdisk.dat}.
\item \shellvar{\$LOCAL}\file{/}\shellvar{\$ARCH} -- an architecture-dependent directory
       containing the executables, libraries, and object code for \PSIthree.
\end{itemize}

An example of a checked-out source tree is found in \file{/home/xerxes/psi3/psi3\_dist},
which we will call \shellvar{\$SRC}, for convenience. \shellvar{\$SRC}
has the following pertinent subdirectories:
\begin{itemize}
\item \shellvar{\$SRC}\file{/lib} -- contains non-architecture-dependent ``library'' data. This
currently includes the psi man page nroff macro file, \file{macro.psi}, the basis
set data, \file{pbasis.dat}, the program execution order, \file{psi.dat}, and tmp disk
information for the local machines (CCQC by default), \file{tmpdisk.dat}. 
\item \shellvar{\$SRC}\file{/include} -- contains architecture-independent header files.
\item \shellvar{\$SRC}\file{/src} -- contains source code for binaries,
libraries, and utilities. \\
\shellvar{\$SRC}\file{/src} has the following subdirectories:
\begin{itemize}
\item \shellvar{\$SRC}\file{/src/util} -- contiains the source code for \module{psipp}, \module{tmpl}.
\item \shellvar{\$SRC}\file{/src/lib} -- contains the source code for all of the \PSIthree\ libraries,
including \library{libpsio}, \library{libipv1}, etc.
\item \shellvar{\$SRC}\file{/src/bin} -- contains the source code for all of the \PSIthree\ 
binaries. 
\end{itemize}
\end{itemize}
