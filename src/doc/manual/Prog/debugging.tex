%
% PSI Programmer's Manual
%
% PSI Debugging Section
%
% Daniel Crawford, 8 February, 1996
%
Debugging \PSIthree\ code using an interactive debugger, such as \file{gdb}
or \file{dbx} can be difficult at times because of the complicated organization of
this large program package.  This section discusses some strategies and
technical details of using such debuggers with the \PSIthree\ code.

\subsection{Code Re-compilation}
Any section of \PSIthree\ code that needs to be debugged must first be re-compiled
with the ``-g'' flag turned on.  This flag is set in the \file{MakeVars}
file in the directory above each module or library's source code directory.
For example, to debug the \PSIcscf\ program, one would first clean the
existing object code out of the \file{~/psi3\_dist/src/\$ARCH/src/bin/cscf}
directory (using \file{make clean}).  Then edit the \file{MakeVars} file
above the \PSIcscf\ source code: set \file{COPT = -g}.  Then re-compile the
module.  If debugging information is needed for a library routine as well,
then follow this same procedure for the library in question.  Technically
only the routines of interest need to be re-compiled, though it is
frequently more convenient to simply re-compile the entire library or
module.

\subsection{Multiple Source Code Directories}
The most difficult problem of debugging \PSIthree\ code is that object code and
source code generally reside in separate directories.  This allows one to
maintain objects for several achitectures simultaneously.  In addition,
library codes are kept separate from binary (module) codes.  Most
interactive debuggers allow the programmer to specify multiple source code
search directories using simple command-line options.  For example, if one
were debugging the \PSIcscf\ program, and needed access to the
\library{libciomr.a} library source code in addition to that of \PSIcscf, one
could use \file{gdb}'s ``dir'' command to search several source code
directories:
\begin{verbatim}
dir ~/psi3_dist/src/bin/cscf ~/psi3_dist/src/lib/libciomr
\end{verbatim}
Additionally, such commands can be placed in the user's \file{.gdbinit}
file.  In \file{dbx} the ``use'' command specifies multiple source directories.
More recent versions of the \file{gdb} debugger can determine the full path
name of each source file compiled with a \file{-g} flag.  This is very
convenient and {\em makes \file{gdb} the preferred debugger in general}.

\PSIthree\ Fortran modules can be significantly more difficult to debug.  This is
because the Fortran source code (\file{.F} suffix) must be preprocessed by
\module{psipp} before it is passed to the compiler (see section \ref{psipp}).
Additionally, the preprocessed files (\file{.f} suffix) are usually
automatically deleted when the compilation stops due to the \file{make}
program's defaults (all such intermediate files are deleted).  So, if the
programmer tries to use the unprocessed code with the debugger, line
numbers won't match and basically he or she will be wasting his or her
time!  To keep the preprocessed source files, {tt make} must be told that
they are important; add the following line to the \file{Makefile} in the
Fortran module's object code directory:
\begin{verbatim}
.PRECIOUS: %.f
\end{verbatim}
This is a GNU \file{make} directive which prevents deletion of all
intermediate \file{.f} files when the compilation stops.  Then add the
current directory to the debugger's source directory search path, and plow
ahead.  Note, however, that any changes made to \file{.f} files must be
duplicated the corresponding unprocessed source file to be permanent.
