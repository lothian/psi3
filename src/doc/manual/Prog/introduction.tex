%
% PSI Programmer's Manual
%
% Introduction
%
% Daniel Crawford, 24 January, 1996
%


The purpose of this manual is to provide a reasonably detailed overview of
the source code and programming philosophy of \PSIthree, such that programmers
interested in contributing to the code will have an easier task.  In
section \ref{Fundamental_PSI}, the essential elements of a C-language \PSIthree\ 
program are discussed, with emphasis on the input parsing and I/O
functions.  Section \ref{Other_Libs} provides documentation of a number of
other important libraries, including the library of functions for reading from the
checkpoint file, \library{libfile30.a}, the Quantum Trio function library,
\library{libqt.a}, and the ``integrals with labels'' function library,
\library{libiwl.a}.  Section \ref{PSI_Fortran} is an overview of the structure
of the few remaining Fortran-based \PSIthree\ modules (e.g. \module{intder95}),
including details of the memory allocation procedures.  Section \ref{CVS} gives an
introduction to the source control system used to maintain the code at the
CCQC, section \ref{Style} offers advice on appropriate programming style for
\PSIthree\ code, and section \ref{Debugging} gives some suggestions on interactive
debugging PSI C and Fortran code. The appendices contain
important reference material, including the currently accepted \PSIthree\ 
citation, format information for some of the most important text files used
by \PSIthree\ modules, advice on Makefile construction for both developmental and
production-level \PSIthree\ code, and complete type declarations for functions in
each supported library.
