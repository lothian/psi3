\begin{verbatim}
=====================================================================
COMBINATIONS

double combinations(int n, int k)

   Function finds the number of combinations of n objects taken k
at a time (i.e. "n choose k").  Returns a double because the number
might be very large.

Parameters:
   n   =  number of objects in total
   k   =  number of objects taken at a time

Returns:
   number of combinations of n objects taken k at a time
   ("n choose k").  Result is returned as double.


=====================================================================
FACTORIAL

double factorial(int n)

   Function returns n factorial (i.e. n!)

Parameters:
   n  = number to take factorial of

Returns:
   n factorial, as a double word (since n! can get very large).


=====================================================================
FILL_SYM_MATRIX

void fill_sym_matrix(double **A, int size)

   This function fills a symmetric matrix by placing the elements of 
the lower triangle into the upper triangle.

Parameters:
   A    = matrix to symmetrize
   size = number of rows or columns in matrix (assume square)

Returns: 
   none


=====================================================================
INVERT_MATRIX

void invert_matrix(double **a, double **y, int N)

   The function takes the inverse of a matrix using the C routines
in Numerical Recipes in C (available in libciomr).

Parameters:
   a  = matrix to take the inverse of (modified by invert_matrix())
   y  = the inverse matrix
   N  = the size of the matrices

Returns:
   none

Note: The original matrix is modified by invert_matrix()


=====================================================================
MAT_IN

int mat_in(FILE *fp, double **array, int width, int max_length, 
      int *stat)

   Function to read in a matrix.  Simple version for now.

Parameters:
   fp         =  file pointer to input stream
   array      =  matrix to hold data
   width      =  number of columns to read
   max_length = maximum number of rows to read
   stat       = pointer to int to hold status flag (0=ok, 1=error)

Returns: 
   number of rows read
   Also modifies stat to = error code (0 = ok, 1 = error)


=====================================================================
NORMALIZE

void normalize(double **A, int rows, int cols)

   Function to normalize a set of vectors.  Assume the vectors to
be normalized are in the ROWS of matrix A.

Parameters:
   A     =  matrix (of doubles) to be normalized
   rows  =  number of rows in matrix
   cols  =  number of columns in matrix

Returns:
   none


=====================================================================
REORDER_QT

void reorder_qt(int *docc, int *socc, int *frozen_docc,
      int *frozen_uocc, int *order, int *orbs_per_irrep, int nirreps)

   Obtain the reordering array for Quantum Trio Standard Ordering.
This function returns an integer array (in 'order') which maps 
a basis function in Pitzer ordering (orbitals grouped according to
irrep) and gives the corresponding index in the Quantum Trio ordering
scheme, which orders the orbitals as: frozen core, docc, socc, 
uocc, and frozen uocc.  Within each subspace, orbitals are ordered
according to irrep.

Parameters:
   docc            = number of doubly occupied orbitals per irrep
   socc            = number of singly occupied orbs per irrep
   frozen_docc     = number of frozen doubly occ'd orbs per irrep
   frozen_uocc     = number of frozen unocc'd orbs per irrep
   order           = array to hold the reordering map
   orbs_per_irrep  = number of orbitals per irrep
   nirreps         = total number of irreps in point group

Returns:
   none


=====================================================================
REORDER_RAS

void reorder_ras(int *docc, int *socc, int *frozen_docc,
      int *frozen_uocc, int *order, int *orbs_per_irrep, 
      int *ras1, int *ras2, int *ras3, int *ras4, int do_ras4,
      int nirreps)

   Obtain the reordering array for RAS wavefunctions.  This function 
returns an integer array (in 'order') which maps a basis function 
in Pitzer ordering (orbitals grouped according to irrep) and gives 
the corresponding index in the RAS scheme, which orders the 
orbitals as: frozen core, RAS1, RAS2, RAS3, [RAS4], frozen uocc.
Within the RAS subspaces, the orbitals are arranged according to
docc, then socc, then uocc.  Within this ordering, orbitals are
further ordered according to irrep.

Parameters:
   docc            = number of doubly occupied orbitals per irrep
   socc            = number of singly occupied orbs per irrep
   frozen_docc     = number of frozen doubly occ'd orbs per irrep
   frozen_uocc     = number of frozen unocc'd orbs per irrep
   order           = array to hold the reordering map
   orbs_per_irrep  = number of orbitals per irrep
   ras1            = number of RAS1 orbitals per irrep
   ras2            = number of RAS2 orbitals per irrep
   ras3            = number of RAS3 orbitals per irrep (if do_ras4=0
                     then this array will be overwritten by a 
                     routine that calculates it automatically)
   ras4            = if do_ras4=0, pass NULL or an empty array.
                     if do_ras4=1, pass an empty array (ras4 will
                     be automatically calculated)
   do_ras4         = 1 to use a RAS4 subspace, 0 otherwise
   nirreps         = total number of irreps in point group

Returns:
   none


=====================================================================
SCHMIDT

void schmidt(double **A, int rows, int cols, FILE *outfile)

   Gram-Schmidt orthogonalize a set of vectors.  Assume we're 
orthogonalizing the ROWS, since in C a vector is usually a row more 
often than a column.

Parameters:
   A    = matrix to orthogonalize (matrix of doubles)
   rows = rows of A
   cols = columns of A
outfile = pointer to output file

Returns:
   none


=====================================================================
SCHMIDT_ADD

   Assume A is a orthogonal matrix.  This function Gram-Schmidt
orthogonalizes a new vector v and adds it to matrix A.  A must 
contain a free row pointer for a new row.  Don't add orthogonalized 
v' if norm(v') < NORM_TOL.

int schmidt_add(double **A, int rows, int cols, double *v)

Parameters:
   A    = matrix to add new vector to
   rows = current number of rows in A (A must have ptr for 
          'rows+1' row.)
   cols = columns in A
   v    = vector to add to A after it has been made orthogonal 
          to rest of A

Returns:
   1 if a vector is added to A, 0 otherwise


=====================================================================

 POPLE(): Uses Pople's method to iteratively solve linear equations
          Ax = b

void pople(double **A, double *x, int dimen, int num_vecs, 
           double tolerance, FILE *outfile, int print_lvl)

 Arguments:
 A         = matrix
 x         = initially has vector b, but returns vector x.
 dimen     = dimension of vector x.
 num_vecs  = number of vectors x to obtain.
 tolerance = cutoff threshold for norm of expansion vector.
 outfile   = output file
 print_lvl = Print level for a variety of print options.

 Returns: none

======================================================================
\end{verbatim}