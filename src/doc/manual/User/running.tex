\subsection{Geometry Specification}
Full molecular geometry has to be specified in form of Cartesian coordinates or
a Z-matrix. Cartesian coordinates of atoms are specified via keyword
\keyword{geometry}:
\begin{verbatim}
  geometry = (
    (atomname1 x1 y1 z1)
    (atomname2 x2 y2 z2)
    (atomname3 x3 y3 z3)
        ..........
    (atomnamen xn yn zn)
  )
\end{verbatim}
where \keyword{atomnamei} can be either an element symbol or a full

\subsection{Basis Sets}
\PSIthree\ default basis sets are located in \pbasisdat\ which is located in
{\tt \$psipath/share}. Table X lists basis sets defined in \pbasisdat.
Needless to say, most of them are not commonly used:

"Insert a table here"

Table 2: Summary of Basis sets defined in \PSIthree.

To make a custom basis set, enter the information in either
\basisdat\ or \inputdat. A contracted Cartesian Gaussian-Type Orbital
\begin{eqnarray}
\phi_{\rm CGTO} & = & x^ly^mz^n\sum_i C_i \exp(-\alpha_i[x^2+y^2+z^2])
\end{eqnarray}
where
\begin{eqnarray}
L & = & l+m+n
\end{eqnarray}
is written as
\begin{verbatim}
basis: (
  ATOM_NAME: "BASIS_SET_LABEL" = (
    (L (C1  alpha1)
       (C2  alpha2)
       (C3  alpha3)
       ...
       (Cn  alpha4))   
    )
  )
\end{verbatim}

\subsection{Geometry Optimization}
\PSIoptking\ is the program responsible for orchestrating the
process of geometry optimization. It can do a number of tasks
automatically, such as generating internal coordinates, produce
empirical force constant matrix, if necessary, update it, and
check if geometry optimization is over. Some or all of the following
files are necessary to perform a geometry optimization with \PSIoptking:
\begin{itemize} 
\item \FILE{11.dat} - contains the cartesian geometry and the nuclear forces, produced by \PSIcderiv ;
\item \fconstdat - contains force constants; if absent - empirical force constants will
be generated by \PSIoptking ;
\item \intcodat - contains internal coordinates in a format readable by a human;
if absent - internal coordinates are generated automatically by \PSIoptking .
\end{itemize}

The procedure for setting up such a calculation is as follows: 
\begin{itemize}
\item define geometry 
\item determine occupations if desired
\item define internal coordinates if desired (if unsure - do not)
\item obtain a set of force constants in an fconst.dat file (if unsure - do not) 
\item run a first derivative with the \keyword{opt} flag set to true and \keyword{ nopt} around 5
\end{itemize}
Precision with which geometry is optimized depends on the residual
forces on the nuclei. By default \PSIoptking\ will terminate the job
if the residual cartesian gradients in \FILE{11.dat} are less than
$10^{-5}$ in atomic units. It is probably enough for most
tasks. Going below this will most likely waste CPU
time unless you are doing benchmarks.

An important aspect of a geometry optimization is the accuracy
of the first derivatives of energy that \PSIthree\ computes.
Depending on how poorly your wavefunction has been convereged, the
gradients themselves may not be sufficiently accurate for
the requested convergence criterion. After computing first
derivatives of the energy, \PSIcints\ runs a simple check
of the quality of the energy derivative. It's a good idea to
look at \PSIcints ' output to make sure that the gradients are OK. 

Let us take a look at each step involved in optimizing molecular geometry.

\subsubsection{How to specify geometry}
See documentation for \PSIinput .

\subsubsection{Specifying Electronic Configuration}
The reference electronic configuration of a molecule is specified
via a combination of keywords \keyword{reference} and \keyword{multiplicity}
and occupation vectors \keyword{docc} and \keyword{socc}. Although the latter
may not be necessary as \PSIcscf\ may guess occupations for you
had \keyword{charge}, \keyword{multiplicity}, and \keyword{reference} have been specified.

\PSIthree\ is only capable of recognizing the electronic structure
for molecules in $D_{\rm 2h}$ and its subgroups. To determine the
electronic occupations in \PSIthree, first construct symmetry
orbitals using group theory and fill them according to regular
valence bond arguments. To define your occupations in \PSIthree,
use the \keyword{docc} and \keyword{socc} arrays. But only \keyword{docc}
and \keyword{socc} may not be enough to specify precisely
the spin couplings in your system. That's where \keyword{reference} and
\keyword{multiplicity} keywords come in. \keyword{multiplicity} is equal
2S+1, where S is the spin quantum number of the system. \keyword{reference} can equal 
\begin{itemize}
\item rhf (default) - spin-restricted reference for closed shell molecules.
\keyword{multiplicity} may only equal to 1 in this case.
\item rohf - spin-restricted reference for open shell molecules.
If multiplicity=1 and socc has two singly occupied orbitals in
different symmetry blocks - it's equivalent to the old opentype=singlet statement.
Otherwise it's assumed to be a high-spin open-shell case
(equivalent to the old opentype=highspin statement).
\item uhf - spin-unrestricted reference for closed shell or high-spin (parallel spins) open shell system.
\item twocon for two determinantal wavefunctions. The largest
component should be specified by the docc and socc arrays.
Multiplicity has to be set to 1.
\end{itemize}
For $^1{\rm A}_1$ methylene, the occupation is (1a1)2(2a1)2(1b2)1(3a1)2 so the docc is: 
\begin{verbatim}
reference = rhf or uhf
multiplicity = 1
docc = (3 0 0 1)
\end{verbatim}
For the $^3{\rm B}_1$ state of methylene,
the electronic configuration is (1a1)2(2a1)2(1b2)2(3a1)1(1b1)1 so the docc and socc arrays are: 
\begin{verbatim}
reference = rohf or uhf
multiplicity = 3
docc = (2 0 0 1)
socc = (1 0 1 0)
\end{verbatim}
For the $^1{\rm B}_1$ state of methylene however, the docc and socc arrays are also: 
\begin{verbatim}
reference = rohf
multiplicity = 1
docc = (2 0 0 1)
socc = (1 0 1 0)
\end{verbatim}
Since most of the basis sets are highly contracted in the core regions,
core electrons are routinely frozen and corresponding virtual
orbitals are deleted. This is accomplished via the \keyword{frozen\_docc}
and \keyword{frozen\_uocc} arrays. Simply specify the symmetry of
the frozen orbital and \PSIthree\ will do the rest. 

To freeze the lowest $a_1$ orbital and delete the corresponding
highest $b_2$ orbital, they would look like this: 
\begin{verbatim}
frozen_docc = (1 0 0 0)
frozen_uocc = (0 0 0 1)
\end{verbatim}

\subsubsection{Internal Coordinates and Structure of \keyword{intco} Vector}
This section is largely obsolete now with the addition of the \PSIoptking\ 
program which can generate internal coordinates automatically. At present
\PSIoptking\ cannot handle molecules larger than a few
atoms but it should change in very near future. Hence 
you may still specify internal coordinates manually
as described here, but this ability may become obsolete someday.

\PSIthree\ currently carries out all optimizations in
internal coordinates. The internals are specified in either
\inputdat\ or \intcodat. First, the primitive internals
are defined. These are individual
stretches, bends, torsion, out-of-plane deformations,
and two different linear bends denoted lin1 and lin2.
All of these are defined in Wilson, Decius, and Cross. An example for methane is
below:
\begin{verbatim}
intco: (
   stre = (
     (1 1 2)
     (2 1 3)
     (3 1 4)
     (4 1 5)
   )
   bend = (
     (5 2 1 5)
     (6 3 1 5)
     (7 4 1 5)
     (8 2 1 4)
     (9 3 1 4)
     (10 2 1 3)
   )
\end{verbatim}
After the primitives are defined, they are constructed into
symmetrized internals with the totally symmetric placed in
the SYMM vector and the rest placed in the ASYMM vector. For
optimizations, only the SYMM internals need to be defined.
Likewise, if during an optimization a molecule breaks symmetry,
the internals have been improperly defined. Again, methane is done
below:
\begin{verbatim}
    symm = (
     ("(1) stretch"(1 2 3 4))
   )
    asymm = (
     ("(2) E bend"(10 7))
     ("(3) T2 stretch"(1  2  -3  -4 ))
     ("(4) T2 bend"(10 -7))
     ("(5) E torsion"(8 -5  -9  6))
     ("(6) T2 bend" (8 -6))
     ("(7) T2 bend"( 5 -9))
     ("(8) T2 stretch"(1  3  -2  -4))
     ("(9) T2 stretch"(1  4 -2  -3))
   )
)
\end{verbatim}
The SYMM and ASYMM vectors have two or three components: the first
is a label enclosed by quotation marks and the second is the list
of primitive internals comprising this vector. Some
internals have been multiplied by -1 to reflect the appropriate
symmetries. If the internals need to be weighted by some prefactor,
then a third vector may be used: 
\begin{verbatim}
    symm = (
     ("generic coord" (1 -2 -3) (2.0 1.0 1.0))
   )
\end{verbatim}
For more information in defining symmetric internals, refer to Cotton's text.

\subsubsection{Force Constant Matrix and Structure of \fconstdat}
The quality of the force constants, or Hessian, is critical for
optimizing weakly bound structures. In order to start an optimization,
one needs the \fconstdat\ file. For those of you that can speak
Fortran 77, this file is written in 8F10.7 format. It is the
lower triangle of the force constant matrix in internal coordinates. The order
of the forces is identical to the order of the SYMM and ASYMM vectors.
For the methane-water dimer, an excerpt from a real \fconstdat\ is shown below: 
\begin{verbatim}
  5.654908
   .217027  5.616085
  -.006096  -.001145  6.078154
  -.055291   .026921   .023732   .317485
   .004063  -.155489  -.003880  -.170201   .873999
  -.146900  -.285108   .001153  -.023346   .196008   .605239
   .001037  -.001959   .002945  -.024597   .014432   .005302   .388622
\end{verbatim}
Ideally, your diagonal elements should be the much larger than
the non-diagonal elements. If you need an \fconstdat\ file, you have four options: 
\begin{enumerate}
\item Create a diagonal matrix of 1's 
\item Create a diagonal matrix with 5 for stretching coordinates,
2 for bending coordinates, and 1 for all other coordinates 
\item Let \PSIoptking\ generate an empirical Hessian for you
\item Run a second derivative to obtain a Cartesian Hessian
and transform that to internals( fconst.dat) with intder. 
\end{enumerate}
Clearly, the list starts at the most
approximate and gets more accurate. 

\subsection{Frequency Analysis}
Currently, analytic second derivatives are not avaliable for any
method in \PSIthree. Hence, for methods with analytic gradients,
one must carry out finite difference calculations. If one has the misfortune of needing frequencies
for methods where only energies exist(e.g. FCI or MP2-R12), finite
differences with gradients approximated by energy
points must be used. There are two types of frequency analysis
programs available within \PSIthree, \PSInormco\ and \PSIintder.
For more details regarding these programs, see the manual pages for each
program respectively.

\subsection{Property evaluation}
For now, take a look at the available documentation for \PSIoeprop .

\subsection{Utilities}
\subsubsection{\PSIgeom}
The program \PSIgeom\ reads a set of Cartesian coordinates
and determines from them the bond distances (Bohr and
angstrom), bond angles, torsional angles, out-of-plane
angles (optional), moments of inertia, and rotational constants.
It requires either a \FILE{11.dat} or \geomdat\ and writes \geomout. 
