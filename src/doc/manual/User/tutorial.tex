\subsection{Running \PSIthree}
The \PSIthree\ suite of codes is built around a modular design which 
allows it great power and flexibility. Any module can be run 
independently (provided suitable datafiles, of course). There also exists a
master program, appropriately called \PSIdriver, which will parse 
an input file, recognize the overall calculation desired, 
and run all the necessary modules in the correct order. To begin with 
\begin{verbatim}
  c4-20:> man psi
\end{verbatim}
will get the online manual page for the \PSIdriver\ master program.
Other man pages for the \PSIdriver\ code can be retrieved likewise.

The only required file is to run \PSIthree\ is \inputdat. 
Unfortunately, no good and simple description of this
file exists. It is rather simple though, and essentially
free format. Examples can be found in
almost every man page. It should be fairly easy to
understand all the examples using common sense. 

\subsection{Format of \inputdat}
Some keys to remember are the following: 
\begin{itemize}
\item Input parsing is case insensitive; once parsed, everything is treated as upper case. 
\item Anything following a percent mark, i.e. \%, is commented out up until the next carriage return. 
\item White space of more than a single space is ignored. 
\item Anything between double quotes, e.g. "/usr/c4/sw/psi", 
is considered one token; there is no change of case, and special 
characters and white space are maintained as part of the token 
but  otherwise ignored. 
\end{itemize}
Input data types are the following: 
\begin{itemize}
\item String : a character sequence 
\item Integer : a \celem{sizeof(integer)} byte integer datum 
\item Real : a \celem{sizeof(double)} byte real datum 
\item Boolean : \keyword{yes}, \keyword{true}, \keyword{1} -- these three values are equivalent;
\keyword{no}, \keyword{false}, \keyword{0} -- these three values are equivalent 
\end{itemize}
Input parameters come in only a few flavors: 
\begin{itemize}
\item \keyword{Keywords = value}, e.g. \keyword{convergence = 12} 
\item \keyword{Vectors = (value1 value2 ... valuen)}, e.g. \keyword{docc = (2 0 1 1)}
\item \keyword{Arrays = ((i1 i2 ... in) (j1 j2 ... jn) ... (n1 n2 ... nn))}
Note the proper number of opening and closing parentheses. Arrays can 
conceptually be any depth/dimension but in practice never go
much beyond level 2. The elements of vectors and arrays 
need not be of the same data type. 
\end{itemize}

Segmented Keywords are joined by a colon, i.e. ":". 
\begin{verbatim}
  scf:convergence = 12
  gugaci:convergence = 8
\end{verbatim}
Segmented keywords with common initial segments can be joined in vectors. 
\begin{verbatim}
  scf:(convergence = 12
       docc = (2 0 1 1))
\end{verbatim}
Note the equivalence of the following three examples with the preceding one. 
\begin{verbatim}
  scf:(convergence = 12 docc = (2 0 1 1)
  scf:( convergence=12  docc=(2 0 1 1))
  scf: (
     convergence = 12 
     docc = (2 0 1 1)
  )
\end{verbatim}

A "default" token can be used as a wild card, to fill any initial 
token field. However, explicit tokens will always override default
tokens if both are present. For instance, a typical keyword search
priority would be: First 
\begin{verbatim}
  module_id:parameter
\end{verbatim}
and then 
\begin{verbatim}
  default:parameter
\end{verbatim}
Thus the two following examples will achieve the same effect when
the module \PSItransqt\ is run, but not when \PSIcscf\ is run. 
\begin{verbatim}
  default: (
    convergence = 8
    docc = (2 0 1 1)
  )
  scf: (
    convergence = 12
  )

  default: (
    convergence = 8
    docc = (2 0 1 1)
  )
  scf: (
    convergence = 10
  )
\end{verbatim}

Ok, enough of all that, you've got the basics down,
the rest can be learned by doing. The smallest \inputdat\
file you would probably ever want to use, and this only as
a learing experience, is: 
\begin{verbatim}
  % file input.dat
  psi: (
    check = true
    )
  % end file input.dat
\end{verbatim}
If we run psi now, what happens? 
\begin{verbatim}
  c4-20:> psi
                      The Psi Execution Driver
   ERROR: a problem arose while reading the required string valued keyword 'WFN'
  c4-20:>
\end{verbatim}
Obviously we need to add a \keyword{WFN}. So 
\begin{verbatim}
  % file input.dat
  psi: (
    check = true
    wfn = scf          % for instance
    )
  % end file input.dat

  c4-20:> psi
                      The Psi Execution Driver
   WFN       = SCF
   DERTYPE   = NONE
   REFERENCE = RHF
   CHECK     = YES
   
   'CHECK' is YES, so nothing will be executed.
   The following programs would otherwise be executed:
   cints
   cscf
  c4-20:>
\end{verbatim}
As you can see, some defaults are chosen. We are now doing a
RHF SCF Energy Point. First the module (or program) \PSIcints\ will
run to calculate the integrals, then \PSIcscf\ will run to calculate the
RHF SCF energy. 

Should we run this, just for the learning experience? Sure,
but first we need to do a couple things. Like find out what
files and input parameters are needed by \PSIcints\ and \PSIcscf. 
\begin{verbatim}
  c4-20:> man cints
  c4-20:> man cscf
\end{verbatim}
\PSIcints\ needs \inputdat\ and \FILE30\ and will generate \FILE33,
\FILE35, \FILE36, and \FILE37. It requires only 2 input parameters
and the defaults look ok. \PSIcscf\ needs \inputdat, \FILE30, \FILE33,
\FILE35, \FILE36, and \FILE37 and
a few input parameters which we will come back to. 

So what's this \FILE30? It is the basic binary working file for the \PSIthree\
quantum chemistry package. It contains all the juicy details of your job,
such as geometry, basis set, occupation scheme, etc.
Since no one likes to write binary files,
we will create \FILE30 with the program \PSIinput.
For more information on this program 
\begin{verbatim}
  c4-20:> man input
\end{verbatim}


\subsection{Setting up a Calculation}
I've decided that we'll start with STO water for our example, and
after reading the man pages described above I come up with the following: 
\begin{verbatim}
  % file input.dat
  default: (

    label = "water STO HF energy point"
    memory = (8.0 Mbytes)
    wfn = scf
    reference = rhf

    files: (
      default: (
        name = "h2osto"
        nvolume = 4
        volume1 = "/tmp1/psiuser"
        volume2 = "/tmp2/psiuser"
        volume3 = "/tmp2/psiuser"
        volume4 = "/tmp2/psiuser"
        )
      file30: (
        nvolume = 1
        volume1 = "./"
        )
      )
    )

  psi: (
    check = true
    )

  input: (
  %    note that all atoms are specified now,
  %    not only the symmetry unique portion
    basis = sto-3g
    units = angstrom
    zmat = (
      (o)
      (h 1 0.9600)
      (h 1 0.9600 2 104.5)
      )
    )
  % end file input.dat
\end{verbatim}
You may wonder about the memory flag I've tucked in the default. As far
as I know, it's undocumented, but can be set for some PSI modules.
Acceptable units are REAL, INTEGER, BYTES,
KBYTES, or MBYTES. Generally it is a good practice to keep
\keyword{default:memory} small, and increase \keyword{program\_id:memory} as necessary. 

Also, the files section popped up out of nowhere. That just tells
the Psi modules how to handle all the scratch files. It is properly documented by: 
\begin{verbatim}
  c4-20:> man files
\end{verbatim}
You will want to make sure that you have the directories specified by the \keyword{volume\#}
parameter on the machine on which you are running. Oh, and use your own username
not "\keyword{psiuser}" in the last field. 
\begin{verbatim}
  c4-20:> input
\end{verbatim}
You should now have two new files \outputdat\ and \keyword{h2osto.30}.
Look at the first one, and it should make sense. The second
is the previously mentioned \FILE30, but do not bother to look
at it; it is binary. 

Going on, 
\begin{verbatim}
  c4-20:> psi
\end{verbatim}
reminds us that we will be running \PSIcints\ and \PSIcscf. We have everything
we need for the first, but how about the second. 
\begin{verbatim}
  c4-20:> man cscf
\end{verbatim}
Well, there are alot of possibilities, but the defaults are generally
sound, so we really only need to add in our occupation scheme for
the doubly-occupied orbitals. Or do we? The fact is
\PSIcscf\ can guess occupations for you using
core Hamiltonian orbital eigenvalues, but, as with any computer
program, you have to be cautious. Let us be adventurous here and
let the program guess.
In general, always make sure you are computing the state you
want to and not the state that the program chose. 

Notice at this point that I am leaving out the
majority of the file \inputdat. It is getting a little big
to repetitiously include it within this document. 

\subsection{Carrying Out a Calculation}
Assuming you have done everything detailed in the previous
section, we can run an actual job and get a number out. 
\begin{verbatim}
  c4-20:> psi
\end{verbatim}
Ahhhh. We forgot to change the \keyword{psi:check=true} flag,
so nothing was really run. Change it to \keyword{psi:check=false},
or simply comment it out (since the default value is false) by inserting a "\keyword{ \%}" at
the start of that line, and run psi again. Take a look at the file \outputdat\
and become familiar with it. 

We now have the basic \inputdat\ file. Only minor modifications will allow
it to be used to run a great variety of jobs. First
of all let's move on to another basis set. The only thing that needs to be
changed is the parameter \keyword{input:basis=sto-3g} to,
for instance, \keyword{input:basis=dz}. But for consistency,
why not change \keyword{default:label} and \keyword{default:files:default:name}? 
\begin{verbatim}
  c4-20:> input
  c4-20:> psi
\end{verbatim}
Whenever you change the basis parameter, you must run input so that
the change in basis is included into \FILE30. Very important.
Changing \keyword{default:files:default:name} every time you
change the basis parameter is a good habit, as it will
insure that you run \PSIinput. Do you see why? 

An aside about \PSIcscf\ and guess wavefunctions: when you run \PSIcscf,
it (by default, but can be overridden) automatically checks
to see if there is an old wavefunction in \FILE30\ which can be used
as a guess wavefunction, and if so, it uses it.
Each time \PSIcscf\ completes, (converged or not) it writes
the final wavefunction to \FILE30. Each time \PSIinput\ is run, it overwrites
\FILE30\ and any wavefunction that might have been in there is
lost. Nothing tricky about any of this, but just something to
keep in mind. Currently there is no wavefunction expander
to make a guess wavefunction from that for a smaller basis set. 

Well, energy points are well and good, but without optimizations,
we are not going to get very far. So, we need \keyword{psi:dertype=first},
\keyword{psi:opt=true}, and \keyword{psi:nopt=7} to run up to 7 cycles of geometry
optimization (Or we could set \keyword{default:dertype=first}
and \keyword{default:opt=true}, or even mix and match). Set \keyword{psi:check=true}
and run \PSIdriver\ to see what modules will be run. Then check the man
pages for each module and see what you need for it to run. For instance,
in this case we see that \PSIcderiv\ will be run. {\tt man \PSIcints}
shows that us that everything is set OK. \PSIoptking\ is the
last program to be run in each optimization cycle. \PSItwo\
users will appreciate the ease with which it can optimize molecules
as it can generate internal coordinates for you as well as guess a
force constant matrix. Hence defaults are sufficient to run this program.

OK! Go ahead and run \PSIdriver. In the 5th cycle \PSIoptking\ 
should return a non-zero value and \PSIdriver\ will stop the
procedure. It will indicate that the optimization is
over. Let's up the ante and go for an optimization with a DZP basis set.

You know the procedure for improving the basis set.
Just switch the label, the files name and the basis parameter.
Oh, before I forgot, you have to remove file named \optaux. Then run \PSIinput\ 
and \PSIdriver. The job runs fine. Go ahead and look at \outputdat.
You may also wish to look at \FILE{11.dat}. \fconstdat\ contains now
an improved force constant matrix. \PSIoptking\ will automatically
use this improved FC matrix if it is present.

OK, the calculation did converge in 4 steps. What if it didn't?
Just increasing the number of optimization steps might just
do that but might not. In general, 10 optimization steps should be enough
for anyone. If it doesn't converge by then,
your initial guess geometry or initial FC matrix
or both were bad. Rethink your situation.
Oops, this is supposed to be a howto-psi, not a QC-theology. 
