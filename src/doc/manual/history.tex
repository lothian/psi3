%
% History of Psi
%
% Daniel Crawford, 24 January, 1996
%

The PSI suite of {\em ab initio} quantum chemistry programs is the result
of an ongoing attempt by a cadre of graduate students and postdoctoral
associates to produce efficient and fast computer code.  Perhaps the first
contribution to what is now referred to as ``PSI'' was written in 1976 by
Robert Lucchese, who was an undergraduate in the Schaefer group at the
University of California, Berkeley. Unfortunately, the direct configuration
interaction code he constructed no longer exists.  The first piece of code
which could still be found in the second version of the package (\PSItwo)
was written by Bernie Brooks in 1977-78; this is the well-known \module{gugaci}
 configuration interaction energy code.
Later, in 1978 after Prof.~Russ Pitzer provided the integrals
program (which eventually became the old \module{intsth} module), the
package became known as the BERKELEY suite.  In 1987, after the Schaefer
research group moved to the Center for Computational Quantum Chemistry here
at the University of Georgia, the code was renamed PSI. In 1999, 
extensive effort of many has come to its conclusion in what we now call
\PSIthree\ -- a PSI suite with a completely new face. \PSIthree\ is fundamentally
superior to older incarnasions of PSI, it has lost many limitations of \PSItwo,
but much work is still needed to make \PSIthree\ a mainstream research tool.
Today the package has the capability to determine wavefunctions, energies, analytic
gradients, and various molecular properties based on a variety of theories,
including spin-restricted Hartree-Fock(RHF), spin-restricted open-shell
Hartree-Fock (ROHF), configuration interaction (CI) (including a variety of
multireference CI's and full CI), coupled-cluster (CC) including CC with
variationaly optimized orbitals, M{\o}ller-Plesset
perturbation theory (MPPT) including explicitly correlated second-order
M{\o}ller-Plesset energy (MP2-R12), and complete-active-space self-consistent field
(CASSCF) theory. The work in progress includes Kohn-Sham Density Function Theory (DFT)
energy and gradients, efficient automated geometry
optimization and frequency analysis by finite differences, parallelization of
some portions of the code for execution on shared- and distributed-memory
parallel machines, and many other exciting projects.
