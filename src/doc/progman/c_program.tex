%
% PSI Programmer's Manual
%
% Essentials of a C Program
%
% David Sherrill, 31 January 1996
% Updates by TDC, 2002.
%

To function as part of the PSI package, a program must incorporate
certain required elements.  This section will discuss the header
files, global variables, and functions required to integrate a C
program into \PSIthree.  Figure \ref{fig:Essential_C_Program} presents
a minimal \PSIthree\ program, whose elements are described below.

\begin{figure}
\begin{verbatim}
                #include <stdio.h>
                #include <libipv1/ip_lib.h>
                #include <libpsio/psio.h>
                #include <libciomr/libciomr.h>

                FILE *infile, *outfile;

                int main(void)
                {
                  extern char *gprgid(void);

                  ffile(&infile,"input.dat",2);
                  ffile(&outfile,"output.dat",1);
                  ip_set_uppercase(1);
                  ip_initialize(infile,outfile);
                  ip_cwk_add(":DEFAULT");
                  ip_cwk_add(gprgid());
                  psio_init();

                  /* to start timing, tstart(outfile); */
                
                  /* Insert code here */

                  /* to end timing, tstop(outfile); */

                  psio_done();
                  fclose(infile);
                  fclose(outfile);
                  ip_done();
                }

                char *gprgid(void)
                {
                   char *prgid = ":CODE_NAME";
                   return(prgid);
                }               
\end{verbatim}
\caption{The essential elements of a \PSIthree\ C-language program.}
\label{fig:Essential_C_Program}
\end{figure}

The required include files are \file{libipv1/ip\_lib.h},
\file{libciomr/libciomr.h}, \file{libpsio/psio.h}, and of course
\file{stdio.h}.  The first of these is for the Input Parser Library,
Version 1 (\file{libipv1.a}), which is described in section
\ref{C_IP}.  The second file contains function prototypes for the C
Math Routines and old-style I/O library, \file{libciomr.a}.  The third
file analogously provides clean interface to functions of the new C
I/O system described in section \ref{C_IO_New}.  The PSI libraries
require that \celem{infile} and \celem{outfile} be global variables.
(Note that work to allow input and output files named something other
than ``input.dat'' and ``output.dat'', respectively, is underway.)

The functions in \celem{main()} are required to set up and shut down
the input parser; they will be described in detail in section
\ref{C_IP}.  Even if the program will not do any input parsing, it is
necessary to include these statements, because some of the
functionality of the input parser is assumed by \library{libciomr.a}
and \library{libpsio.a}.  For instance, opening a binary file via
\celem{psio\_open()} requires parsing the \keyword{files} section of
\inputdat\ so that a unit number (e.g.~52) can be translated into a
filename (e.g. \file{h2o.52}).  Timing information (when the program
starts and stops, and how much user, system, and wall-clock time it
requires) can be printed to the output file by adding calls to
\celem{tstart()} and \celem{tstop()} (from \library{libciomr.a}).

The sole purpose of the simple function \celem{gprgid()} is to provide
the input parser a means to determine the name of the current program.
This allows the input parser to add the name of the program to the
input parsing keyword tree.  This function is used by
\library{libpsio.a}, though the functionality it provides is rarely
used.

NB: The library \library{libciomr.a} contains older I/O functions that
have been superceded by functions in \library{libpsio.a}.  However,
you are encouraged to use the many non-I/O functions in
\library{libciomr.a}.
