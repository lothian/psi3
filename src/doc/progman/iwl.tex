The library \library{libiwl.a} contains functions for reading and
writing to files with the "Integrals With Labels" (IWL) format created
by David Sherrill in 1994, modeled after the format of the old
integrals file from \PSItwo.  Most functions deal with four-index
quantitites, but there are also a few which deal with two-index
quantities such as one-electron integrals.  The IWL format specifies
that the 4-index quantities are stored on disk in several buffers;
each buffer has a header segment which gives some useful info.
Currently, the header is arranged as follows: one integer word is used
as a flag, telling whether the current buffer is the last buffer in
the file.  The next integer gives the number of integrals (and their
associated labels) in the current buffer.  After this header
information, each buffer contains two data segments: one for labels,
and one for the values of the associated integrals.  The datasize for
the labels is defined using typedefs, so it is easy to change
(currently, it is a short int); likewise for the integral values
(currently of type double).  The length of these data segments is NBUF
* 4 * sizeof(Label) and NBUF * sizeof(Value), respectively.  The
current use of short ints for Label is really somewhat excessive,
making the files somewhat larger than strictly necessary.  However,
this avoids confusing bit-packing schemes, and instantly allows us to
have up to something like 65,536 basis functions addressable.

The functions previously documented in this manual have been removed
because that documentation is now out of date.  Documentation of the
library is now created directly from the source code using the 
{\tt doxygen} program and is available at
\htmladdnormallink{
{\tt http://vergil.chemistry.gatech.edu/psi/devel/libs/doxygen/html}}
{http://vergil.chemistry.gatech.edu/psi/devel/libs/doxygen/html}.

