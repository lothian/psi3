\PSIthree\ is capable of computing energies, analytic gradients, and
linear response properties using a number of coupled cluster models.
Table \ref{table:ccsummary} summarizes these capabilities.  This
section describes how to carry out coupled cluster calculations within
\PSIthree.
\begin{table}
\begin{center}
\begin{tabular}{cccccc}
\hline
\hline
Reference & Method & Energy    & Gradient  &  Exc. Energies & LR Props \\
\hline
RHF       & CCSD    & Y & Y & Y & Y  \\
RHF       & CCSD(T) & Y & N & --& -- \\
ROHF      & CCSD    & Y & Y & Y & N  \\
ROHF      & CCSD(T) & N & N & --& -- \\
UHF       & CCSD    & Y & Y & Y & N  \\
UHF       & CCSD(T) & Y & N & --& -- \\
Brueckner & CCD     & Y & N & N & N  \\
Brueckner & CCD(T)  & Y & N & --& -- \\
\hline
\hline
\end{tabular}
\end{center}
\caption{Current coupled cluster capabilities of \PSIthree.}
\label{table:ccsummary}
\end{table}

\subsection{Ground-State Energies}

To compute a ground-state CCSD or CCSD(T) energy at a fixed geometry,
the following keywords are required:
\begin{itemize}
\item {\tt wfn = ccsd}, {\tt ccsd\_t}, {\tt bccd}, or {\tt bccd\_t}.
\item {\tt reference = rhf}, {\tt rohf}, or {\tt uhf}.
\item {\tt jobtype = sp}
\end{itemize}

