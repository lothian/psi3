\section{Introduction} \label{introduction}

\subsection{Overview}
This manual explains how to use the
\PSIthree\ suite of {\em ab initio} quantum chemical programs.  
In this section, we provide an overview of some
of the features of \PSIthree\ along with the prerequisite steps for
running calculations.  Section \ref{tutorial} provides a brief
tutorial to help new users get started.  Section \ref{input} offers
further details into the structure of \PSIthree\ input files
and a discussion of some of the most important options.
Later sections deal with the different types of computations which
can be done using \PSIthree\ (e.g., Hartree-Fock, MP2, coupled-cluster)
and general procedures such as geometry optimization
and vibrational frequency analysis.  
The appendix includes a description of
the input keywords and command-line options for each module, as well
as numerous examples of \PSIthree\ input and basis set files.
[{\em CDS: Except there's nothing there yet... We can dispense with the
full list of options for now since we still have the man pages.  Some
additional examples might be nice even though we have the test case 
directory.}]
For the latest \PSIthree\ documentation, check 
\htmladdnormallink{{\tt www.psicode.org}}
{http://www.psicode.org/}.

The \PSIthree\ package was developed to perform high-accuracy 
quantum mechanical computations on challenging chemical species
and to provide an infrastructure for the development of new
theoretical techniques.  Hence, it has a very flexible input
scheme which allows non-standard computations, and it is easily
adapted to enable new capabilities.

\subsection{Obtaining and Installing \PSIthree}
\label{installation}

The latest version of the \PSIthree\ program package may be obtained
at \htmladdnormallink{{\tt www.psicode.org}}{http://www.psicode.org}.
The source code is available as a gzipped tar archive (named, for
example, {\tt psi3.X.tar.gz}), and binaries (in {\tt .rpm} format) may
be available for certain architectures.  For detailed installation and
testing instructions, please refer to the the \PSIthree\
Installation Manual, available as part of the package or at the
\PSIthree\ website above.

\subsection{Supported Architectures}
The majority of \PSIthree\ was developed on IBM RS/6000//AIX
and Intel x86/Linux workstations. The complete list of
tested architectures to which \PSIthree\ has
been ported is shown in Table \ref{table:ports}.
\begin{table}[h]
\caption{Platforms on which \PSIthree\ has been installed successfully.}
\label{table:ports}
\begin{center}
\begin{tabular}{ll} \hline\hline
Architecture              &  Notes \\ \hline
Compaq Alpha Tru64 UNIX   & 64-bit mode \\
IBM RS/6000 AIX 4.1-4.3   & 64-bit mode \\
IBM PowerPC AIX 4.3       & 64-bit mode \\
Linux (Intel x86, Athlon, Opteron)  & Linux 2.2-2.4 \\
SGI IRIX64 ($>$6.5.15)    & 64-bit \\ \hline\hline
\end{tabular}
\end{center}
\end{table}
If you don't find your system in the Table, there's a good chance
that you will be able to install \PSIthree\ on your system
if you have the prerequisite math and utility libraries described 
in the installation manual.

\subsection{Capabilities}
\PSIthree\ can perform {\em ab initio} computations employing
basis sets of up to 32768 contracted Gaussian-type functions of
virtually arbitrary orbital quantum number.
\PSIthree\ can recognize and exploit the largest Abelian subgroup of the
point group describing the full symmetry of the molecule.
Table \ref{table:methods} displays the range of theoretical
methods available in \PSIthree .
\begin{table}
\caption{Summary of theoretical methods available in \PSIthree.} \label{table:methods}
\parsep 10pt
\begin{center}
\begin{tabular}{lccc} \hline\hline
Method        & Energy & Gradient & Hessian \\ \hline
RHF SCF       & Y & Y & Y \\
ROHF SCF      & Y & Y & N \\
UHF SCF       & Y & N & N \\
CIS/RPA/TDHF  & Y & N & N \\
TCSCF         & Y & Y & N \\
CASSCF        & Y & Y & N \\
RAS-CI        & Y & N & N \\
RHF MP2       & Y & N & N \\
RHF MP2-R12   & Y & N & N \\
RHF CCSD      & Y & Y & N \\
ROHF CCSD     & Y & Y & N \\ 
UHF CCSD      & Y & Y & N \\
RHF CCSD(T)   & Y & N & N \\
ROHF CCSD(T)  & N & N & N \\
UHF CCSD(T)   & Y & N & N \\
RHF EOM-CCSD  & Y & N & N \\
ROHF EOM-CCSD & Y & N & N \\
\hline\hline
\end{tabular}
\end{center}
\end{table}
Geometry optimization (currently restricted to true minima on the potential
energy surface) can be performed using either analytic gradients
or energy points.  Likewise, vibrational frequencies can be 
computed using analytic second derivatives or by finite
differences of analytic gradients.
\PSIthree\ can also compute an extensive list of one-electron properties.

\subsection{Technical Support}
The \PSIthree\ package is distributed for free and without any guarantee
of reliability, accuracy, suitability for any particular purpose.  
No obligation to provide technical support is expressed or implied.  
As time allows, the developers will attempt to answer inquiries directed
to 
\htmladdnormallink{{\tt psimaster@psicode.org}}{mailto:psimaster@psicode.org}.
For bug reports, specific and detailed information, with example inputs,
would be appreciated.



