\section{Geometry Optimization and Vibrational Frequency Analysis} \label{opt}

\PSIthree\ is capable of carrying out geometry optimizations for a variety of
molecular structures using either analytic and numerical energy gradients.
When analytic gradients are available (see Table \ref{table:methods}),
\PSIthree\ will automatically generate and use so-called redundant
internal coordinates for carrying out the optimization.  For methods
for which only energies are available, \PSIthree\ will use delocalized
internal coordinates to generate geometrical displacements for computing
finite-difference gradients.  For both methods, a guess force-constant
matrix is generated and updated using the BFGS method.

In addition, \PSIthree\ is capable of computing harmonic vibrational
frequencies for a number of different methods using either analytic energy
first or second derivatives.  (At present, only RHF-SCF analytic second
derivatives are available.)  For those methods for which analytic gradients
have been coded (see Table \ref{table:methods}), \PSIthree\ will generate
displaced geometries along symmetrized, delocalized internal coordinates,
compute the appropriate first derivatives, and use finite-difference
methods to compute the Hessian.

For finite-difference procedures for vibrational frequency calculations,
the user should keep in mind that, for geometric displacements along
non-totally-symmetric coordinates, by definition, the molecular point
group symmetry will decrease.  As a result, any \PSIthree\ keywords in
the input file which depend on the number of irreducible representations
will be incorrect.  For example, if the user must specify the {\tt DOCC}
keyword in order to obtain the correct MO occupations for their system,
computation of non-symmetric vibrational modes via finite-differences
will necessarily fail.  The developers are working to further automate
the computation of vibrational frequencies such that the correlation
of irreducible representations between point groups can be handled
correctly.  This feature will be available in a future release of the
package.

The following keywords are pertinent for geometry optimizations and
vibrational frequency analyses:
\begin{description}
\item[JOBTYPE = string]\mbox{}\\
This keyword (described earlier in this manual) must be set to
{\tt OPT} for geometry optimizations and {\tt FREQ} for frequency analyses.
\item[DERTYPE = string]\mbox{}\\
This keyword (also described earlier) must be set to {\tt NONE} is only
energies are available for the chosen method and {\tt FIRST} if analytic
gradients are available.
\end{description}
