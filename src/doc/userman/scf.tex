\subsection{Hartree-Fock Self-Consistent-Field} \label{scf}
                                                                                
Hartree-Fock molecular orbital theory forms the cornerstone of
ab initio quantum chemistry.  Until the advances in the accuracy
of Kohn-Sham density functional theory in the 1990's, Hartree-Fock
theory was the method of choice for obtaining results for large
molecules without resorting to standard empirical or semiempirical
approaches.  Molecular properties obtained by Hatree-Fock theory are
generally at least qualitatively correct, although they can be
quantitatively poor in many instances.
                                                                                
\PSIthree\ solves the Hatree-Fock equations in a basis of Gaussian 
functions using an iterative, self-consistent-field (SCF) procedure.  The
final molecular orbitals are those which minimize the energy, 
subject to the electron configuration specified by the user (or
guessed by the program).  The process is continued until the largest
change in an element of the density matrix drops below 10$^{-n}$, where
$n$ is an integer specified by the \keyword{convergence} keyword.

Of course the efficiency of the iterative procedure depends on the choice of
initial guess.  The \PSIcscf\ module will attempt to use previously obtained
orbitals as a guess if they are available.  This can be particularly
advantageous when diffuse functions are present; in that case, it may
be easiest to run the computation with a smaller basis and project
those orbitals onto the larger basis by specifying the \keyword{--chkptmos}
command-line argument or the \keyword{chkpt\_mos=true} keyword in input
when running the \PSIinput\ program for the larger basis.  If
old MO's are not available, \PSIcscf\ uses a core Hamiltonian guess
by default.  The convergence of the SCF procedure is accelerated by Pulay's 
direct inversion of the iterative subspace (DIIS) approach, and it is possible
to modify the behavior of the DIIS through various keywords, 
although this is seldom necessary.  

It is important to point out that the
SCF approach does not rigorously guarantee that the final orbitals
actually correspond to a minimum in orbital space; at convergence,
the only guarantee is that the gradient of the energy with respect 
to orbital rotations is zero: this could be a global minimum, a local
minimum, or a saddle point in orbital rotation space.  While this
is not usually an issue (typically the lowest minimum consistent with 
the electron configuration is found), it can be a problem sometimes for 
radicals, diradicals, bond breaking, or unusual bonding situations.  The
\PSIstable\ module can be used to test for the stability of Hartree-Fock
wave functions.

The most commonly used keywords are found below.  More specialized keywords
are available in the man pages.

\begin{description}
\item[MAXITER = integer]\mbox{}\\
This gives the maximum number of iterations.  The default is 40.
\item[CONVERGENCE = integer]\mbox{}\\
This specifies how tightly the wavefunction will be converged.
Convergence is determined by comparing the RMS change in the
density matrix ("delta P") to the given value.  The convergence
criterion is 10**(-integer).  The default is 7 if both DERTYPE
= NONE and WFN = SCF are given and 10 otherwise.
\item[LEVELSHIFT = real]\mbox{}\\
This specifies the level shift.  The default is 1.
\item[DIRECT = boolean]\mbox{}\\
Specifies whether to do the SCF calculation with an integral direct
technique.  The default is false.
\item[PRINT\_MOS = boolean]\mbox{}\\
Specifies whether to print the molecular orbitals or not.  The default
is false.
\end{description}
