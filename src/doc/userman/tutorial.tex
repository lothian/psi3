\section{A \PSIthree\ Tutorial} \label{tutorial}

Before going into all the details of the \PSIthree\ input format, 
we will begin with a tutorial on how to run some very simple 
computations.

The first thing to point out is that electronic structure programs
like \PSIthree\ make significant use of disk drives.  Therefore, 
for any but the smallest calculations, {\em it is very important for the user 
to be sure that \PSIthree\ is writing its temporary files to a 
disk drive physically attached to the computer running the computation}.
If a user's directory is remotely mounted by NFS, and if \PSIthree\
tries to write its temporary files to this directory, it will slow 
the program and the network down dramatically.  To avoid this
situation, by default \PSIthree\ will write temporary files to 
\file{/tmp}.  This will work for the small examples in this tutorial,
but for real computations you will want to set up a default scratch
file path as described in sections \ref{scratchfile} and \ref{psirc},
since your \file{/tmp} directory may not be large enough.

The \PSIthree\ suite of codes is built around a modular design which 
allows it great power and flexibility. Any module can be run 
independently (provided suitable datafiles, of course). There also exists a
master program, appropriately called \PSIdriver, which will parse 
an input file, recognize the overall calculation desired, 
and run all the necessary modules in the correct order.  As we will
see later, it is possible to define entirely new calculation types simply 
by editing a text file.

By default, the \PSIdriver\ program reads its input from a text file
called \inputdat, and output from \PSIthree\ is written to 
\outputdat.  However, it is also possible to specify different input
and output filenames by adding these to the command line, like this:
\begin{verbatim}
  psi3 input-name output-name
\end{verbatim}

What does a \PSIthree\ input file look like?  First of all, due to
the modular nature of the program, the input may be split into different
sections, each beginning with a keyword and enclosed in parentheses.
For now, we will ignore this flexibility and simply put all of the 
input into a default area called \keyword{psi}:

\begin{verbatim}
% This is a sample PSI3 input file.
% Anything after a percent sign is treated as a comment.

psi: (
  label = "cc-pVDZ SCF H2O"
  jobtype = sp
  wfn = scf
  reference = rhf
  basis = "cc-pVDZ"
  puream = true
  zmat = (
    o
    h 1 0.957
    h 1 0.957 2 104.5
  )
)
\end{verbatim}

As you can see, the input is made up of \keyword{keyword = value} pairs.  
Values can be strings (like \keyword{"cc-pVDZ"}), booleans 
(\keyword{true/false, 1/0, yes/no}), integers, or real numbers.  One can 
also have arrays of data (as in the the \keyword{zmat} entry).  Generally
speaking, input is not case-sensitive.  Anything following a percent sign
is treated as a comment, and white space of more than a single space is
ignored.  

The above input is for a Hartree-Fock SCF (\keyword{wfn = scf})
calculation of water using a cc-pVDZ basis set 
(\keyword{basis = "cc-pVDZ"}).  This is a 
single-point calculation (\keyword{jobtype = sp}) at the experimental 
geometry.  Since the ground state of water is closed shell, restricted 
Hatree-Fock (RHF) orbitals are used (\keyword{reference = rhf}).  
The geometry is specified in the \keyword{zmat} array using the
standard quantum chemistry Z-matrix input format (for beginners, a short
tutorial on Z-matrix inputs is available at
\htmladdnormallink{{\tt www-rcf.usc.edu/$\sim$krylov/AbInitioCourse/zmat.html}}{
http://www-rcf.usc.edu/~krylov/AbInitioCourse/zmat.html}).  Geometries
may also be input in Cartesian coordinates, as we will see later.

The final keyword, \keyword{puream}, tells the program whether or not
to use Cartesian basis functions, or basis functions derived from the
pure angular momentum functions.  For $d$ orbitals, for example, 
there are 6 Cartesian $d$ functions, but only 5 pure angular 
momentum $d$ functions.  Dunning's correlation consistent basis sets
(cc-pVXZ, etc.) were designed to use pure angular momentum functions
(\keyword{puream=true}), and other basis sets may have different 
``standard choices'' for using pure or Cartesian functions.  The \PSIthree\ 
program defaults to \keyword{puream = false} unless told otherwise.

Why are some strings in quotation marks (e.g., \keyword{label = "cc-pVDZ
SCF H2O"}), while others are not (e.g., \keyword{wfn = scf})?  Any
string containing spaces or other special characters like asterisks
must be placed in double quotes.  Anything between double quotes is 
considered one token; there is no change of case, and special characters 
and white space are maintained as part of the token but otherwise ignored.
Since basis sets often contain special characters, it is a good idea to 
specify the basis in quotes always.  For values which don't have to be put 
in double quotes, it never hurts to put them in quotes anyway.

Let's run this calculation.  Assuming the input is given in a file
\file{sp.in}, we can run \PSIthree\ as:
\begin{verbatim}
psi3 sp.in sp.out
\end{verbatim}

This will run the driver program \PSIdriver\, which will read the
input from \file{sp.in} and write the output to \file{sp.out}.  
The driver program will print something like this:
\begin{verbatim}
 The PSI3 Execution Driver
 
PSI3 will perform a RHF SCF energy computation.
 
The following programs will be executed:
 
 input
 cints
 cscf
 
input
cints
cscf
\end{verbatim}
The \PSIdriver\ program figures out what type of calculation it is
(a restricted Hartree-Fock self-consistent-field energy computation)
and it runs the appropriate \PSIthree\ modules, which are 
\PSIinput, \PSIcints, and \PSIcscf.  The \PSIinput\ program reads
the geometry and basis set information from the input file and
places this information in a ``checkpoint file'' (with a filename
ending in \file {.32}) which is used
during the calculation.  The \PSIcints\ module computes the one- and
two-electron integrals, and the \PSIcscf\ module computes the 
SCF energy.  Since this is all we requested, the program stops here.

What does the output of the program look like?  There is quite a lot
in the output, so we will focus only on the most pertinent portions.
The first part should look something like this:
\begin{verbatim}
******************************************************************************
tstart called on aurelius.chemistry.gatech.edu
Tue Aug 26 17:35:10 2003
                                                                                
                                --------------
                                  WELCOME TO
                                    PSI  3
                                --------------
                                                                                
  LABEL       = cc-pVDZ SCF H2O
  SHOWNORM    = 0
  PUREAM      = 1
  PRINT_LVL   = 1
                                                                                
  -Geometry before Center-of-Mass shift (a.u.):
       Center              X                  Y                   Z
    ------------   -----------------  -----------------  -----------------
          OXYGEN      0.000000000000     0.000000000000     0.000000000000
        HYDROGEN      0.000000000000     0.000000000000     1.808467771070
        HYDROGEN      1.750863805261     0.000000000000    -0.452804167853
                                                                                
...
\end{verbatim}

This first section corresponds to the output from the \PSIinput\ program,
which translates the basis set and geometry information from the user's
input file.  At the beginning of each module, timing information is
printed which gives the time and date the module is started and the name
of the computer running the computation.  After parsing the geometry, the
program shifts the molecule so that the origin is at the center of mass,
and it may also rotate the molecule so that any spatial symmetry in the 
molecule may be easier to detect.  The program then prints out the 
detected point group of the molecule, which in this case is C2v:
\begin{verbatim}
  -SYMMETRY INFORMATION:
    Computational point group is C2v
    Number of irr. rep.      = 4
    Number of atoms          = 3
    Number of unique atoms   = 2
\end{verbatim}
You should double-check that the program correctly identifies the symmetry
of your molecule; the \PSIthree\ program is not tolerant of any rounding 
errors in the user input of geometry.

Next, the basis set information is printed out explicitly in a format
identical to that used for customized input of basis sets, as described
in section \ref{custom-basis}.  After this, the program prints a summary
of the basis set information, including the number of basis functions
in atomic orbitals (AO's) and symmetry-adapted orbitals (SO's), and
the number of SO's per irredicible representation (irrep) of the point group:
\begin{verbatim}
  -BASIS SET INFORMATION:
    Total number of shells = 12
    Number of primitives   = 31
    Number of AO           = 25
    Number of SO           = 24
                                                                                
    Irrep    Number of SO
    -----    ------------
      1           11
      2            2
      3            4
      4            7
\end{verbatim}
The number of SO's is very useful to check to verify that the pure
angular momentum setting is correct; for example, using 
6 Cartesian d functions (\keyword{puream=false}) or 5 pure 
angular momentum d functions (\keyword{puream=true}) will 
give the same number of AO's, but a different number of SO's.

After some additional geometry information, the \PSIinput\ program
prints the nuclear repulsion energy and then a list of internuclear
distances:
\begin{verbatim}
    Nuclear Repulsion Energy (a.u.) =       9.196934380445
                                                                                
  -The Interatomic Distances in angstroms:
                                                                                
           1           2           3
                                                                                
    1   0.0000000
    2   0.9570000   0.0000000
    3   0.9570000   1.5133798   0.0000000
\end{verbatim}

The nuclear repulsion energy is another very useful diagnostic; when
comparing the results of two different calculations, if the calculations
are really performed at the same geometry, the nuclear repulsion energies
should match.

Once the \PSIinput\ program completes, a file ending in \file{.32} should
be created.  This is the ``checkpoint file'' which is used to communicate
basic information (the basis set, the geometry, etc) between modules
in \PSIthree.  By default, this file will be placed in the current working
directory when the program is run.  The name and location of the file can
be modified by the user as described later in sections \ref{scratchfiles}
and \ref{psirc}.

Now that \PSIthree\ knows about the molecule and basis set, the next
module to execute is the \PSIcints\ module.  
\begin{verbatim}
                  --------------------------------------------
                    CINTS: An integrals program written in C
                     Justin T. Fermann and Edward F. Valeev
                  --------------------------------------------
                                                                                
                                                                                
  -OPTIONS:
    Print level                 = 1
    Integral tolerance          = 1e-15
    Max. memory to use          = 2500000 double words
    Number of threads           = 1
    LIBINT's real type length   = 64 bit
...
\end{verbatim}
This program computes
integrals over symmetry orbitals.  By default, these integrals are
written to disk unless the user has specified an integrals-direct
computation.  The integrals file can be very large, and its size grows
rapidly with the size of the molecule and basis set.  When \PSIcints\ 
is finished, it prints the number of two-electron integrals:
\begin{verbatim}
    Wrote 11412 two-electron integrals to IWL file 33
\end{verbatim}

Once the integrals have been written, it is time to perform the 
Hartree-Fock SCF computation using the \PSIcscf\ module.  
\begin{verbatim}
             ------------------------------------------
                                                                                
                CSCF3.0: An SCF program written in C
                                                                                
              Written by too many people to mention here
                                                                                
             ------------------------------------------
                                                                                
  I think the multiplicity is 1.
  If this is wrong, please specify the MULTP keyword
                                                                                
  label       = cc-pVDZ SCF H2O
  wfn          = SCF
  reference    = RHF
  multiplicity = 1
  charge       = 0
  direct       = false
  dertype      = NONE
  convergence  = 7
  maxiter      = 40
  guess        = AUTO
                                                                                
  nuclear repulsion energy        9.1969343804448
  first run, so defaulting to core-hamiltonian guess

...
\end{verbatim}

The first few lines print out information about the spin multiplicity
specified by the user, the convergence level, maximum number of iterations,
etc.  After this, the program attempts to figure out the symmetries
of the occupied orbitals using a core Hamiltonian guess.  The \PSIcscf\
program sometimes guesses occupations wrong, especially for highly strained
molecules, radicals, unusual bonding situations, or large basis sets
(particularly with diffuse functions).  Unfortunately, this is a common 
problem with programs which have point-group symmetry built in throughout the 
program; however, the user can override the guess by specifying these
occupations explicitly using the \keyword{docc} and \keyword{socc} keywords.
In this case, the program determines these occupations correctly, as
below.
\begin{verbatim}
  Using core guess to determine occupations
                                                                                
                                                                                
  Symmetry block:   A1    A2    B1    B2
  DOCC:              3     0     1     1
  SOCC:              0     0     0     0
\end{verbatim}
The \keyword{DOCC} array lists number of doubly-occupied orbitals per
irrep, and the \keyword{SOCC} array lists the number of singly-occupied
orbitals per irrep (unnecessary for closed-shell molecules).  These
occupations could have been specified explicitly in user input as:
\begin{verbatim}
docc = (3 0 1 1)
socc = (0 0 0 0)
\end{verbatim}

Once the occupations are available, the self-consistent-field procedure
begins to solve the Hatree-Fock equations.  
\begin{verbatim}
  iter       total energy        delta E         delta P          diiser
    1       -68.8728484552    7.806978e+01    0.000000e+00    0.000000e+00
    2       -71.2986301202    2.425782e+00    2.008202e-01    9.307684e-01
    3       -73.8237872940    2.525157e+00    1.056816e-01    9.960987e-01
    4       -75.8742610925    2.050474e+00    5.205943e-02    7.105479e-01
    5       -76.0217491034    1.474880e-01    1.313776e-02    1.898892e-01
    6       -76.0263470429    4.597939e-03    1.920732e-03    5.155862e-02
    7       -76.0267968846    4.498417e-04    8.594423e-04    1.595832e-02
    8       -76.0268077609    1.087637e-05    1.063540e-04    1.111887e-03
    9       -76.0268081461    3.851364e-07    1.914348e-05    2.419943e-04
   10       -76.0268081747    2.867948e-08    4.819477e-06    6.654983e-05
   11       -76.0268081769    2.160775e-09    1.655548e-06    1.348540e-05
   12       -76.0268081769    1.818989e-11    1.534696e-07    1.509549e-06
   13       -76.0268081769    1.264766e-12    4.735582e-08    4.720895e-07
\end{verbatim}
For each iteration, the program lists the total electronic energy, the
change in energy from the previous iteration, the RMS change in the
density matrix, and the maximum element in the DIIS error array.  Once
the delta P has dropped below $10^{-n}$, where $n$ is the number specified
as the convergence criterion (by default, 7), the self-consistent-field
procedure is considered complete.  

At the end of the computation, the orbital energies are reported (in 
Hartrees), along with their symmetry labels:
\begin{verbatim}
 
Orbital energies (a.u.):
                                                                                
  Doubly occupied orbitals
   1A1    -20.550390     2A1     -1.336826     1B2     -0.699406
   3A1     -0.566646     1B1     -0.493169
                                                                                
                                                                                
  Unoccupied orbitals
   4A1      0.185609     2B2      0.256291     3B2      0.789452
   5A1      0.854627     6A1      1.163500     2B1      1.200384
   4B2      1.253301     7A1      1.444438     1A2      1.476370
   3B1      1.674656     8A1      1.867219     5B2      1.935269
   6B2      2.453470     9A1      2.491061     4B1      3.285961
   2A2      3.339124    10A1      3.510949    11A1      3.866084
   7B2      4.147878

\end{verbatim}
For typical closed-shell molecules, the occupied orbitals should have
negative energies, and the unoccupied orbitals should have positive
energies, as in this case.  The highest occupied orbital (HOMO) is
a 1B1 orbital with an energy of -0.493169 Hartree, and the lowest
unoccupied orbital (LUMO) is the 4A1 orbital with energy 0.185609 Hartree.

Frequently, more specialized input will be required.  For example,
maybe you wish to compute the energy of a radical anion using a large,
diffuse basis set.  It may be difficult to converge the Hatree-Fock 
wavefuction for a case like this, and it might require a different
initial guess, more iterations, etc.  Special keywords for accomplishing
such tasks are described later in this user's manual.  It is also
possible to find this information through the use of \PSIthree\'s 
manual pages and the {\tt man} command.  For the man pages to be accessible,
the user must add them to their {\tt MANPATH} environmental variable (as
described in the installation manual).  For C shell, this can be 
accomplished using
\begin{verbatim}
setenv MANPATH $psipath/doc/man:$MANPATH
\end{verbatim}
where {\tt psipath} is the path to your installed version of 
\PSIthree (e.g., \file{/usr/local/psi3}).  The manual for the \PSIcscf\
module can then be accessed simply by typing
\begin{verbatim}
man cscf
\end{verbatim}
which will display a summary of the module and a list of user options.

