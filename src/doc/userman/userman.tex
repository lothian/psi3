\documentclass[12pt]{article}
\usepackage{html}
\setlength{\textheight}{9in}
\setlength{\textwidth}{6.5in}
\setlength{\hoffset}{0in}
\setlength{\voffset}{0in}
\setlength{\headheight}{0in}
\setlength{\headsep}{0in}
\setlength{\topmargin}{0in}
\setlength{\oddsidemargin}{-0.05in}
\setlength{\evensidemargin}{-0.05in}
\setlength{\marginparsep}{0in}
\setlength{\marginparwidth}{0in}
\setlength{\parsep}{0.8ex}
\setlength{\parskip}{1ex plus \fill}
\baselineskip 18pt
\renewcommand{\topfraction}{.8}
\renewcommand{\bottomfraction}{.2}

%%%%%%%%%%%%%%%%%%%%%%%
% Start document
%%%%%%%%%%%%%%%%%%%%%%%
\setcounter{page}{1}
\begin{document}

%%%%%%%%%%%%%%%%%%%%%%%
% Definitions
%%%%%%%%%%%%%%%%%%%%%%%

\newcommand{\PSItwo}{{\tt PSI2}}
\newcommand{\PSIthree}{{\tt PSI3}}

%
% Psi Modules
%
\def\module#1{{\tt #1}}
\newcommand{\PSIdriver}{\module{psi}}
\newcommand{\PSIinput}{\module{input}}
\newcommand{\PSIcints}{\module{cints}}
\newcommand{\PSIcderiv}{\module{cints --deriv1}}
\newcommand{\PSIdetci}{\module{detci}}
\newcommand{\PSIdetcas}{\module{detcas}}
\newcommand{\PSIdetcasman}{\module{detcasman}}
\newcommand{\PSIclag}{\module{clag}}
\newcommand{\PSIccenergy}{\module{ccenergy}}
\newcommand{\PSIccsort}{\module{ccsort}}
\newcommand{\PSIpsi}{\module{psi}}
\newcommand{\PSIcscf}{\module{cscf}}
\newcommand{\PSIoptking}{\module{optking}}
\newcommand{\PSItransqt}{\module{transqt}}
\newcommand{\PSInormco}{\module{normco}}
\newcommand{\PSIintder}{\module{intder95}}
\newcommand{\PSIgeom}{\module{geom}}
\newcommand{\PSIoeprop}{\module{oeprop}}

%
% Psi Library
%
\def\library#1{{\tt #1}}

%
% Psi and Unix Files
%
\def\FILE#1{{\tt file#1}}
\def\file#1{{\tt #1}}
\newcommand{\inputdat}{\file{input.dat}}
\newcommand{\outputdat}{\file{output.dat}}
\newcommand{\fconstdat}{\file{fconst.dat}}
\newcommand{\intcodat}{\file{intco.dat}}
\newcommand{\optaux}{\file{opt.aux}}
\newcommand{\basisdat}{\file{basis.dat}}
\newcommand{\pbasisdat}{\file{pbasis.dat}}
\newcommand{\geomdat}{\file{geom.dat}}
\newcommand{\geomout}{\file{geom.out}}

%
% Psi Keywords
%
\def\keyword#1{{\tt #1}}

%
% Psi C and Fortran Language elements
%
\def\celem#1{{\tt #1}}
\def\felem#1{{\tt #1}}

%
% Unix stuff
%
\def\unixid#1{{\em #1}} % names of groups and users
\def\shellvar#1{{\tt #1}}

%
% Nice output for function description
%
% Needs 4 arguments: function declaration,
%  description, arguments, and return values
%
% Call \initfuncdesc before using \funcdesc
%
\newcommand{\initfuncdesc}
{\newlength{\lcwidth}
\settowidth{\lcwidth}{Arguments:}
\newlength{\rcwidth}
\setlength{\rcwidth}{\linewidth}
\addtolength{\rcwidth}{-1.0\lcwidth}
\addtolength{\rcwidth}{-6.0\tabcolsep}
}

\newcommand{\funcdesc}[4]{
\celem{#1} \\
#2

\begin{tabular}{lp{\rcwidth}}
Arguments: & #3\\
Returns: & #4
\end{tabular}}


\begin{center}
\ \\
\vspace{2.0in}
{\bf {\Large Installation Manual for the \PSIthree\ Program Package}} \\
\vspace{0.5in}
C\ David Sherrill$^a$ and T.\ Daniel Crawford$^b$ \\
\ \\
{\em $^a$Center for Computational Molecular Science and Technology, \mbox{Georgia 
Institute of Technology,} Atlanta, Georgia 30332-0400} \\
\vspace{0.1in}
{\em $^b$Department of Chemistry, Virginia Tech, Blacksburg, Virginia 24061-0001}
\ \\
\vspace{0.3in}
Version of: \today
\end{center}

\thispagestyle{empty}

\newpage
%\tableofcontents
\newpage

\section{Introduction} \label{introduction}
%\section{Introduction} \label{introduction}

\subsection{Overview} 

This manual explains how to use the \PSIthree\ suite of {\em ab initio}
quantum chemical programs.  In this section, we provide an overview of
some of the features of \PSIthree\ along with the prerequisite steps for
running calculations.  Section \ref{tutorial} provides a brief tutorial to
help new users get started.  Section \ref{input} offers further details
into the structure of \PSIthree\ input files and a discussion of some of
the most important options.  Later sections deal with the different types
of computations which can be done using \PSIthree\ (e.g., Hartree-Fock,
MP2, coupled-cluster) and general procedures such as geometry optimization
and vibrational frequency analysis.  The appendix will eventually include a
description of the input keywords and command-line options for each module,
as well as numerous examples of \PSIthree\ input and basis set files.
For the latest \PSIthree\ documentation, check \htmladdnormallink{{\tt
www.psicode.org}} {http://www.psicode.org/}.

The \PSIthree\ package was developed to perform high-accuracy quantum
mechanical computations on challenging chemical species and to provide an
infrastructure for the development of new theoretical techniques.  Hence,
it has a very flexible input scheme which allows non-standard computations,
and it is easily adapted to enable new capabilities.

The following citation should be used in any publication utilizing the
\PSIthree\ program package:

\begin{quotation}
\noindent
T. Daniel Crawford, C. David Sherrill, Edward F. Valeev, Justin
T. Fermann, Rollin A. King, Matthew L. Leininger, Shawn T. Brown,
Curtis L. Janssen, Edward T. Seidl, Joseph P. Kenny, and Wesley D. Allen,
{\em J. Comput. Chem.}, in press.

\end{quotation}

\subsection{Obtaining and Installing \PSIthree}
\label{installation}

The latest version of the \PSIthree\ program package may be obtained at
\htmladdnormallink{{\tt www.psicode.org}}{http://www.psicode.org}.  The
source code is available as a gzipped tar archive (named, for example, {\tt
psi3.X.tar.gz}), and binaries may be available for certain architectures.
For detailed installation and testing instructions, please refer to the
the \PSIthree\ Installation Manual, available as part of the package or
at the \PSIthree\ website above.

\subsection{Supported Architectures}
The majority of \PSIthree\ was developed on IBM RS/6000/AIX
and x86/GNU Linux workstations. The complete list of
tested architectures to which \PSIthree\ has
been ported is shown in Table \ref{table:ports}.
\begin{table}[h]
\caption{Platforms on which \PSIthree\ has been installed successfully.}
\label{table:ports}
\begin{center}
\begin{tabular}{ll} \hline\hline
Architecture              &  Notes \\ \hline
Compaq Alpha Tru64 UNIX   & 64-bit mode \\
IBM AIX 4.3.3, 5.x on PowerPC & 64-bit mode \\
Linux on Intel/AMD x86, AMD x86-64  & 32 and 64-bit\\
Apple OS X (Darwin) on PowerPC & \\
SGI IRIX64 ($>$6.5.15)    & 64-bit \\ \hline\hline
\end{tabular}
\end{center}
\end{table}
If you don't find your system in the Table, there's a good chance
that you will be able to install \PSIthree\ on your system
if you have the prerequisite tools and math and utility libraries described 
in the installation manual.

\subsection{Capabilities}
\PSIthree\ can perform {\em ab initio} computations employing
basis sets of up to 32768 contracted Gaussian-type functions of
virtually arbitrary orbital quantum number.
\PSIthree\ can recognize and exploit the largest Abelian subgroup of the
point group describing the full symmetry of the molecule.
Table \ref{table:methods} displays the range of theoretical
methods available in \PSIthree .

\begin{table}
\caption{Summary of theoretical methods available in \PSIthree.} \label{table:methods}
\parsep 10pt
\begin{center}
\begin{tabular}{lccc} \hline\hline
Method           & Energy & Gradient & Hessian \\ \hline
RHF SCF          & Y & Y & Y \\
ROHF SCF         & Y & Y & N \\
UHF SCF          & Y & N & N \\
HF DBOC          & Y & N & N \\
CIS/RPA/TDHF     & Y & N & N \\
TCSCF            & Y & Y & N \\
CASSCF           & Y & Y & N \\
RAS-CI           & Y & N & N \\
RAS-CI DBOC      & Y & N & N \\
RHF/UHF/ROHF MP2     & Y & N & N \\
RHF MP2-R12      & Y & N & N \\
RHF/UHF/ROHF CCSD    & Y & Y & N \\
RHF/UHF/ROHF CCSD(T) & Y & N & N \\
RHF EOM-CCSD     & Y & N & N \\
ROHF EOM-CCSD    & Y & N & N \\
\hline\hline
\end{tabular}
\end{center}
\end{table}
Geometry optimization (currently restricted to true minima on the potential
energy surface) can be performed using either analytic gradients
or energy points.  Likewise, vibrational frequencies can be 
computed using analytic second derivatives or by finite
differences of analytic gradients.
\PSIthree\ can also compute an extensive list of one-electron properties.

\subsection{Technical Support} The \PSIthree\ package is
distributed for free and without any guarantee of reliability,
accuracy, suitability for any particular purpose.  No obligation
to provide technical support is expressed or implied.  As time
allows, the developers will attempt to answer inquiries directed to
\htmladdnormallink{{\tt crawdad@vt.edu}}{mailto:crawdad@vt.edu}.
For bug reports, specific and detailed information, with example
inputs, would be appreciated.  Questions or comments regarding
this user's manual may be sent to \htmladdnormallink{{\tt
sherrill@gatech.edu}}{mailto:sherrill@gatech.edu}.





\section{Obtaining and Installing \PSIthree} \label{installation}
%%
% The PSI Installation Manual
%

\documentclass[12pt]{article}
\usepackage{html}
\setlength{\textheight}{9in}
\setlength{\textwidth}{6.5in}
\setlength{\hoffset}{0in}
\setlength{\voffset}{0in}
\setlength{\headheight}{0in}
\setlength{\headsep}{0in}
\setlength{\topmargin}{0in}
\setlength{\oddsidemargin}{-0.05in}
\setlength{\evensidemargin}{-0.05in}
\setlength{\marginparsep}{0in}
\setlength{\marginparwidth}{0in}
\setlength{\parsep}{0.8ex}
\setlength{\parskip}{1ex plus \fill}
\baselineskip 18pt
\renewcommand{\topfraction}{.8}
\renewcommand{\bottomfraction}{.2}

\begin{document}

\newcommand{\PSItwo}{{\tt PSI2}}
\newcommand{\PSIthree}{{\tt PSI3}}

%
% Psi Modules
%
\def\module#1{{\tt #1}}
\newcommand{\PSIdriver}{\module{psi}}
\newcommand{\PSIinput}{\module{input}}
\newcommand{\PSIcints}{\module{cints}}
\newcommand{\PSIcderiv}{\module{cints --deriv1}}
\newcommand{\PSIdetci}{\module{detci}}
\newcommand{\PSIdetcas}{\module{detcas}}
\newcommand{\PSIdetcasman}{\module{detcasman}}
\newcommand{\PSIclag}{\module{clag}}
\newcommand{\PSIccenergy}{\module{ccenergy}}
\newcommand{\PSIccsort}{\module{ccsort}}
\newcommand{\PSIpsi}{\module{psi}}
\newcommand{\PSIcscf}{\module{cscf}}
\newcommand{\PSIoptking}{\module{optking}}
\newcommand{\PSItransqt}{\module{transqt}}
\newcommand{\PSInormco}{\module{normco}}
\newcommand{\PSIintder}{\module{intder95}}
\newcommand{\PSIgeom}{\module{geom}}
\newcommand{\PSIoeprop}{\module{oeprop}}

%
% Psi Library
%
\def\library#1{{\tt #1}}

%
% Psi and Unix Files
%
\def\FILE#1{{\tt file#1}}
\def\file#1{{\tt #1}}
\newcommand{\inputdat}{\file{input.dat}}
\newcommand{\outputdat}{\file{output.dat}}
\newcommand{\fconstdat}{\file{fconst.dat}}
\newcommand{\intcodat}{\file{intco.dat}}
\newcommand{\optaux}{\file{opt.aux}}
\newcommand{\basisdat}{\file{basis.dat}}
\newcommand{\pbasisdat}{\file{pbasis.dat}}
\newcommand{\geomdat}{\file{geom.dat}}
\newcommand{\geomout}{\file{geom.out}}

%
% Psi Keywords
%
\def\keyword#1{{\tt #1}}

%
% Psi C and Fortran Language elements
%
\def\celem#1{{\tt #1}}
\def\felem#1{{\tt #1}}

%
% Unix stuff
%
\def\unixid#1{{\em #1}} % names of groups and users
\def\shellvar#1{{\tt #1}}

%
% Nice output for function description
%
% Needs 4 arguments: function declaration,
%  description, arguments, and return values
%
% Call \initfuncdesc before using \funcdesc
%
\newcommand{\initfuncdesc}
{\newlength{\lcwidth}
\settowidth{\lcwidth}{Arguments:}
\newlength{\rcwidth}
\setlength{\rcwidth}{\linewidth}
\addtolength{\rcwidth}{-1.0\lcwidth}
\addtolength{\rcwidth}{-6.0\tabcolsep}
}

\newcommand{\funcdesc}[4]{
\celem{#1} \\
#2

\begin{tabular}{lp{\rcwidth}}
Arguments: & #3\\
Returns: & #4
\end{tabular}}


\begin{center}
\ \\
\vspace{2.0in}
{\bf {\Large Installation Manual for the \PSIthree\ Program Package}} \\
\vspace{0.5in}
T.\ Daniel Crawford,$^a$ C.\ David Sherrill,$^b$ and Edward F.\ Valeev$^{a}$ 
\\ \  \\
{\em $^a$Department of Chemistry, Virginia Tech, Blacksburg, 
Virginia 24061-0001} \\
\vspace{0.1in}
{\em $^b$Center for Computational Molecular Science and Technology, 
\mbox{Georgia Institute of Technology,} Atlanta, Georgia 30332-0400} \\
\vspace{0.1in}
\ \\
\vspace{0.3in}
\PSIthree\ Version: \PSIversion \\
Created on: \today
\end{center}

\thispagestyle{empty}

\newpage
\section{Compilation Prerequisites}

The following external software packages are needed to complile \PSIthree:
\begin{itemize}
\item C, C++, and FORTRAN77 compilers. The FORTRAN77 compiler is only
  used to determine the symbol-naming convention of and some system
  routines for the BLAS and LAPACK libraries on some architectures. It
  is optional in a few cases (e.g. Mac OS X systems).
\item A well-optimized basic linear algebra subroutine (BLAS) library
  for vital matrix-matrix and matrix-vector multiplication
  routines. (See recommendations below.)
\item The linear algebra package (LAPACK).  \PSIthree\ makes use of
  LAPACK's eigenvalue/eigenvector and matrix inversion routines.  (See
  recommendations below)
\item POSIX threads (Pthreads) library
\item Perl interpreter (version 5.005 or higher)
\item Various GNU utilies: \htmladdnormallink{{\tt
www.gnu.org}}{http://www.gnu.org}
\begin{itemize}
\item {\tt autoconf (version 2.52 or higher)}
\item {\tt make}
\item {\tt flex}
\item {\tt bison}
\item {\tt fileutils} (esp.\ {\tt install})
\end{itemize}
\item For documentation only:
\begin{itemize}
\item {\tt LaTeX}
\item {\tt LaTeX2html} (v0.99.1 or 1.62, including the patch supplied in
psi3/misc)
\end{itemize}
\end{itemize}

\section{Brief Summary of Configuration, Compilation, and Installation}

A good directory for the \PSIthree\ source code is /usr/local/src/psi3.
The directory should {\em not} be named {\tt /usr/local/psi}, as that is
the default installation directory unless changed by the {\tt --prefix}
directive (see below).  It should also not have any periods in the path,
e.g., {\tt /usr/local/psi3.2}, because of a bug in {\tt dvips} which will
cause the compilation of documentation to fail.

The following series of steps will configure and build the \PSIthree\
package and install the executables in /usr/local/psi/bin:

\begin{enumerate}
\item {\tt cd \$PSI3} (your top-level \PSIthree\ source directory)
\item {\tt mkdir objdir}
\item {\tt cd objdir}
\item {\tt ../configure} (may need some of the options below, esp.~if
  {\tt blas} or {\tt lapack} are in non-standard locations)
\item {\tt make}
\item {\tt make tests} (optional, but recommended)
\item {\tt make install}
\item {\tt make doc} (optional)
\end{enumerate}

\noindent
You may need to make use of one or more of the following options to
the {\tt configure} script:
\begin{itemize}
\item {\tt -}{\tt -prefix=directory} --- Use this option if you wish to
  install the \PSIthree\ package somewhere other than the default
  directory, {\tt /usr/local/psi}.  This directory will contain
  subdirectories with the final installed binaries, libraries, 
  documentation, and shared data files.
\item {\tt -}{\tt -with-cc=compiler} --- Use this option to specify a
  C compiler.  One should use compilers that generate reentrant code,
  if possible.  The default search order for compilers is: {\tt cc\_r} (AIX
  only), {\tt gcc}, {\tt icc}, {\tt cc}.
\item {\tt -}{\tt -with-cxx=compiler} --- Use this option to specify a
  C++ compiler.  One should use compilers that generate reentrant
  code, if possible. The default search order for compilers is: {\tt xlC\_r}
  (AIX only), {\tt g++}, {\tt c++}, {\tt icpc}, {\tt cxx}.
\item {\tt -}{\tt -with-fc=compiler} --- Use this option to specify a
  Fortran-77 compiler, which is used to determine linking coventions
  for BLAS and LAPACK libraries and to provide system routines for
  those libraries.  Note that no fortran compiler is necessary on Mac
  OS X systems (see below).  The default search order for compilers
  is: {\tt xlf\_r} (AIX only), {\tt gfortran}, {\tt g77}, {\tt ifort},
  {\tt f77}, {\tt f2c}.
\item {\tt -}{\tt -with-f77-symbol=value} --- This option allows manual
  assignment of the F77 symbol convention, which is necessary for C
  programs to link Fortran-interface libraries such as BLAS and
  LAPACK. This option should only be used by experts and even then
  should almost never be necessary.  Allowed values are:
\begin{itemize}                            
\item[lc] lower-case
\item[lcu]lower-case with underscore (default)
\item[uc] upper-case
\item[ucu] upper-case with underscore
\end{itemize}
\item {\tt -}{\tt -with-ld=linker} --- Use this option to specify
  a linker program. The default is {\tt ld}.
\item {\tt -}{\tt -with-ranlib=ranlib} --- Use this option to specify
  a ranlib program. The default behavior is to detect an appropriate
  choice automatically.
\item {\tt -}{\tt -with-ar=archiver} --- Use this option to specify an
  archiver.  The default is to look for {\tt ar} automatically.
\item {\tt -}{\tt -with-ar-flags=options} --- Use this option to specify
  archiver command-line flags. The default is {\tt r}.
\item {\tt -}{\tt -with-incdirs=directories} --- Use this option to
  specify extra directories where to look for header
  files. Directories should be specified prepended by {\tt -I},
  i.e. {\tt -Idir1 -Idir2}, etc. If several directories are specified,
  enclose the list with single right-quotes, e.g., {\tt -}{\tt
    -with-incdirs='-I/usr/local/include -I/home/psi3/include'}.
\item {\tt -}{\tt -with-libs=libraries} --- Use this option to specify
  extra libraries which should be used during linking. Libraries
  should be specified by their full names or in the usual {\tt -l}
  notation, i.e. {\tt -lm /usr/lib/libm.a}, etc.  If several libraries
  are specified, enclose the list with single right-quotes, e.g., {\tt
    -}{\tt -with-libs='-lcompat /usr/local/lib/libm.a'}.
\item {\tt -}{\tt -with-libdirs=directories} --- Use this option to
  specify extra directories where to look for libraries. Directories
  should be specified prepended by {\tt -L}, i.e. {\tt -Ldir1 -Ldir2},
  etc. If several directories are specified, enclose the list with
  single right-quotes, e.g., {\tt -}{\tt
    -with-libdirs='-L/usr/local/lib -I/home/psi3/lib'}.
\item {\tt -}{\tt -with-blas=library} --- Use this option to specify a
  BLAS library.  If your BLAS library has multiple components, enclose
  the file list with single right-quotes, e.g., {\tt -}{\tt
    -with-blas='-lf77blas -latlas'}.  Note that many BLAS libraries
  can be detected automatically.
\item {\tt -}{\tt -with-lapack=library} --- Use this option to specify
  a LAPACK library.  If your LAPACK library has multiple components,
  enclose the file list with single right-quotes, e.g., {\tt -}{\tt
    -with-lapack='-llapack -lcblas -latlas'}.  note that many LAPACK
  libraries can be detected automatically.
\item {\tt -}{\tt -with-max-am-eri=integer} --- Specifies the maximum
  angular momentum level for the primitive Gaussian basis functions
  when computing electron repulsion integrals.  This is set to
  $g$-type functions (AM=4) by default.
\item {\tt -}{\tt -with-max-am-deriv1=integer} --- Specifies the maximum
  angular momentum level for first derivatives of the primitive
  Gaussian basis functions.  This is set to $f$-type functions (AM=3)
  by default.
\item {\tt -}{\tt -with-max-am-deriv2=integer} --- Specifies the maximum
  angular momentum level for second derivatives of the primitive
  Gaussian basis functions.  This is set to $d$-type functions (AM=2)
  by default.
\item {\tt -}{\tt -with-max-am-r12=integer} --- Specifies the maximum
  angular momentum level for primitive Gaussian basis functions used
  in $r_{12}$ explicitly correlated methods.  This is set to $f$-type
  functions (AM=3) by default.
\item {\tt -}{\tt -with-debug=yes/no} --- This option turns on debugging
  options.  This is set to {\tt no} by default.
\item {\tt -}{\tt -with-opt=options} --- Turn off compiler
  optimizations if {\tt no}.  This is set to {\tt yes} by default.
\item {\tt -}{\tt --with-strict=yes} -- Turns on strict compiler warnings.
\end{itemize}

\section{Detailed Installation Instructions}

This section provides detailed instructions for compiling and
installing the \PSIthree\ package.  

\subsection{Step 1: Configuration}

First, we recommend that you choose for the top-level {\tt \$PSI3}
source directory something other than {\tt /usr/local/psi}; your {\tt
  \$HOME} directory or {\tt /usr/local/src/psi3} are convenient
choices.  Next, in the top-level {\tt \$PSI3} source directory you've
chosen, first run {\tt autoconf} to generate the configure script from
{\tt configure.ac}.  It is best to keep the source code separate from
the compilation area, so you must choose a subdirectory for
compilation of the codes.  A simple option is {\tt \$PSI3/objdir},
which should work for most environments.  However, if you need
executables for several architectures, choose more meaningful
subdirectory names.

$\bullet$ The compilation directory will be referred to as {\tt \$objdir}
for the remainder of these instructions.

In {\tt \$objdir}, run the configure script found in the {\tt \$PSI3}
top-level source directory.  This script will scan your system to locate
certain libraries, header files, etc. needed for complete compilation.
The script accepts a number of options, all of which are listed above.
The most important of these is the {\tt --prefix} option, which selects the
installation directory for the executables, the libraries, header files,
basis set data, and other administrative files.  The default {\tt -}{\tt -prefix}
is {\tt /usr/local/psi}.

$\bullet$ The configure script's {\tt -}{\tt -prefix} directory will be referred
to as {\tt \$prefix} for the remainder of these instructions.

\subsection{Step 2: Compilation}

Running {\tt make} (which must be GNU's {\tt 'make'} utility) in {\tt
\$objdir} will compile the \PSIthree\ libraries and executable
modules.

\subsection{Step 3: Testing}

To execute automatically the ever-growing number of test cases after
compilation, simply execute "make tests" in the {\tt \$objdir}
directory.  This will run each (relatively small) test case and report
the results.  Failure of any of the test cases should be reported to
the developers at \PSIemail. By default, any such failure will stop
the testing process.  If you desire to run the entire testing suit
without interruption, execute "make tests TESTFLAGS='-u -q'". Note
that you must do a "make testsclean" in {\tt \$objdir} to run the test
suite again.

\subsection{Step 4: Installation}

Once testing is complete, installation into \$prefix is accomplished by
running {\tt make install} in {\tt \$objdir}.   Executable modules are
installed in {\tt \$prefix/bin}, libraries in {\tt \$prefix/lib} and basis 
set data and other control strctures {\tt \$prefix/share}.

\subsection{Step 5: Documentation}

If your system has the appropriate utilities, you may build the package
documentation from the top-level {\tt \$objdir} by running {\tt make doc}.  
The resulting files will appear in the {\tt \$prefix/doc} area.

\subsection{Step 6: Cleaning}

All compilation-area object files and libraries can be removed to save
disk space by running {\tt make clean} in {\tt \$objdir}.

\subsection{Step 7: User Configuration}

After the \PSIthree\ package has been successfullly installed, the user will
need to add the installation directory into their path.  If the package
has been installed in the default location {\tt /usr/local/psi3}, then
in C shell, the user should add something like the following to 
their {\tt .cshrc} file:
\begin{verbatim}
setenv PSI /usr/local/psi3
set path = ($path $PSI/bin)
setenv MANPATH $PSI/doc/man:$MANPATH
\end{verbatim}
The final line will enable the use of the \PSIthree\ man pages.
\begin{verbatim}
\end{verbatim}

\section{Recommendations for BLAS and LAPACK Libraries}

Much of the speed and efficiency of the PSI3 programs depends on the
corresponding speed and efficiency of the available BLAS and LAPACK
libraries (especially the former).  In addition, the most common
compilation problems involve these libraries.  Users may therefore
wish to consider the following BLAS and LAPACK recommendations when
building PSI3:

\begin{itemize}
\item It is NOT wise to use the stock BLAS library provided with many
  Linux distributions like RedHat.  This library is usually just the
  netlib ({http://netlib.org/}distribution and is completely
  unoptimized.  PSI3's performance will suffer if you choose this
  route.  The choice of LAPACK is less critical, and so the
  unoptimized netlib distribution is acceptable.  If you do choose to
  use the RedHat/Fedora stock BLAS and LAPACK, be aware that some
  RPM's do not make the correct symbolic links.  For example, you may
  have {\tt /usr/lib/libblas.so.3.1.0} but not {\tt
    /usr/lib/libblas.so}.  If this happens, create the link as, e.g.,
  {\tt ln -s /usr/lib/libblas.so.3.1.0 /usr/lib/libblas.so}.  You may
  need to do similarly for lapack.

\item Perhaps the best choices for BLAS are Kazushige Goto's
  hand-optimized BLAS ({\tt
    http://www.tacc.utexas.edu/resources/software/}) and ATLAS ({\tt
    http://math-atlas.sourceforge.net/}).  These work well on nearly
  every achitecture to which the PSI3 developers have access.  On Mac
  OS X systems, however, the {\tt vecLib} package that comes with
  Xcode works well.

\item PSI3 does not require a Fortran compiler, unless the resident
  BLAS and LAPACK libraries require Fortran-based system libraries.
  If you see compiler complaints about missing symbols like "{\tt
    do\_fio}" or "{\tt e\_wsfe}", then your libraries were most likely
  compiled with g77 or gfortran, which require {\tt -lg2c} to resolve
  the Fortran I/O calls.  Use of the same gcc package for PSI3 should
  normally resolve this problem.

\item The PSI3 configure script can conveniently identify and use
  several different BLAS and LAPACK libraries, but its ability to do
  this automatically depends on a number of factors, including
  correspondence between the compiler used for PSI3 and the compiler
  used to build BLAS/LAPACK, and placement of the libraries in
  commonly searched directories, among others.  PSI3's configure
  script will find your BLAS and LAPACK if any of the the following
  are installed in standard locations (e.g. {\tt /usr/local/lib}):

\begin{itemize}  
    \item ATLAS: {\tt libf77blas.a} and {\tt libatlas.a}, plus netlib's
    {\tt liblapack.a}
    \item MKL: {\tt libmkl.so} and {\tt libmkl\_lapack64.a} (with the Intel compilers)
    \item Goto: {\tt libgoto.a} and netlib's {\tt liblapack.a}
    \item Cray SCSL (e.g. on SGI Altix): {\tt libscs.so} (NB: No Fortran compiler
      is necessary in this case, so {\tt -}{\tt -with-fc=no} should work.)
    \item ESSL (e.g. on AIX systems): {\tt libessl.a}
    \end{itemize}  
  \item If configure cannot identify your BLAS and LAPACK libraries
    automatically, you can specify them on the command-line using the
    {\tt -}{\tt -with-blas} and {\tt -}{\tt -with-lapack} arguments
    described above.  Here are a few examples that work on the PSI3
    developers' systems:
  
    (a) Linux with ATLAS:
  
    {\tt -}{\tt -with-blas='-lf77blas -latlas'} {\tt -}{\tt -with-lapack='-llapack -lcblas'}

    (b) Mac OS X with vecLib: 
  
    {\tt -}{\tt -with-blas='-altivec -framework vecLib'} {\tt -}{\tt -with-lapack=' '}
  
    (c) Linux with MKL and {\tt icc/icpc/ifort}: 
  
    {\tt -}{\tt -with-libdirs=-L/usr/local/opt/intel/mkl/8.0.2/lib/32} {\tt -}{\tt -with-blas=-lmkl} {\tt -}{\tt -with-lapack=-lmkl\_lapack32}
\end{itemize}

\section{Miscellaneous architecture-specific notes}
\begin{itemize}

\item Linux on x86 and x86\_64:
  \begin{itemize}
   \item {\tt gcc} compiler: versions 3.2, 3.3, 3.4, 4.0, and 4.1 have been tested.
   \item Intel compilers: version 9.0 has been tested. We do not recommend
   using version 8.1.
   \item Portland Group compilers: version 6.0-5 has been tested.
   \item Some versions of RedHat/Fedora Core RPM packages for the 
   BLAS and LAPACK libraries fail to make all the required symlinks.  
   For example, you may have {\tt /usr/lib/libblas.so.3.1.0} but not
   {\tt /usr/lib/libblas.so}.  If this happens, create the link as, e.g.,
   {\tt ln -s /usr/lib/libblas.so.3.1.0 /usr/lib/libblas.so}.  You
   may need to do similarly for lapack.
  \end{itemize}

\item Linux on Itanium2 (IA64):
  \begin{itemize}
   \item Intel compilers version 9.0 have been tested and work. Version 8.1
   does not work.
   \item {\tt gcc} compilers work.
  \end{itemize}

\item Mac OS 10.$x$:

  \begin{itemize}
  \item The compilation requires a developer's toolkit (Xcode) from
    {\tt apple.com}.  Note that a fortran compiler is not needed for
    PSI 3.3 on Mac OS X systems.

  \item The {\tt libcompat.a} library is also needed, but it is not
    provided in the Xcode toolkit. If you see compiler complaints
    about missing symbols like {\tt re\_comp} or{\tt re\_exec} then your
    {\tt -lcompat} is missing or PSI3 is not aware of it.  As of 6
    April 2007, it can be obtained from Apple's website at:

{\tt http://www.opensource.apple.com/darwinsource/tarballs/apsl/Libcompat-14.1.tar.gz}

You must sign up for a free developer's account to access the above
library.  You can identify the library to configure by adding {\tt
  -}{\tt -with-libs=-lcompat} to the command line.

\item For apple systems, the latest configure script assumes that the
  {\tt vecLib} will be used for the optimized BLAS and LAPACK
  libraries, unless the user indicates otherwise using the {\tt -}{\tt
    -with-blas} and {\tt -}{\tt -with-lapack} flags to configure.  If
  you encounter difficulty with configure, you may have success
  explicitly indicating the vecLib using:

      {\tt -}{\tt -with-blas='-altivec -framework vecLib'} {\tt -}{\tt
        -with-lapack=' '}

    \item Pre Mac OS 10.4: Certain PSI3 codes require significant
      stackspace for compilation.  Increase your shell's stacksize
      limit before running {\tt make}.  For csh, for example, this is
      done using "unlimit stacksize".  [NB: This limit appears to have
      been lifted starting with Mac OS 10.3.X (Panther).]

  \end{itemize}

\item AIX 4.3/5.$x$ in 64-bit environment:
if IBM VisualAge C++ and IBM XL Fortran are used,
one has to specify manually
the {\tt -q64} compiler flag
that enables production of 64-bit executables.
The following configure options have been tested on an AIX5.2
system with IBM VisualAge C++ 6.0 compiler and IBM XL Fortran 8.1 compiler:
{\tt -}{\tt -with-cc='xlc\_r -q64' -}{\tt -with-cxx='xlC\_r -q64'
 -}{\tt -with-fc='xlf\_r -q64' -}{\tt -with-blas=-lessl
 -}{\tt -with-lapack=<your NETLIB LAPACK library>}. Note that
the reentrant versions of the compilers
are used.

\item SGI IRIX 6.$x$:
  \begin{itemize}
   \item MIPSpro C++ compilers prior to version 7.4 require a command-line flag
   '{\tt -LANG:std}' in order to compile \PSIthree\ properly.

   \item Use command-line flag '{\tt -64}' in order to produce 64-bit \PSIthree\ executables with
   MIPSpro compilers. The following is an example of appropriate configure options:
   \begin{verbatim}
  --with-cc='cc -64' --with-cxx='CC -64 -LANG:std' --with-fc='f77 -64'
   \end{verbatim}

   \item Under IRIX configure will attempt to detect automatically and use
   the optimized SGI Scientific Computing Software Library (SCSL).
  \end{itemize}

\item Compaq Alpha/OSF 5.1: default shell ({\tt /bin/sh})
is not POSIX-compliant which causes some \PSIthree\ makefiles
to fail. Set environmental variable {\tt BIN\_SH} to {\tt xpg4}.

\end{itemize}


\end{document}


\section{A \PSIthree\ Tutorial} \label{tutorial}
%\section{A \PSIthree\ Tutorial} \label{tutorial}

\subsection{Before Getting Started: A Warning about Scratch Files}
Generally, electronic structure programs like \PSIthree\ make
significant use of disk drives.  Therefore, it is very important
to ensure that PSI3 is writing its temporary files to a disk drive
phsyically attached to the computer running the computation.  If it
is not, it will significantly slow down the program and the network.
By default, PSI3 will write temporary files to \file{/tmp}, but you
will want to set up a default scratch path (as described in sections
\ref{scratchfiles} and \ref{psirc}) because the \file{/tmp} directory
is usually not large enough except for small test cases.  In any 
event, you want to be very careful that you are not writing scratch
files to an NFS-mounted directory that is physically attached to a 
fileserver elsewhere on the network.

\subsection{Basic Input File Structure} 

PSI3 reads input from a text file, which can be prepared in any standard
text editor.  The default input file name is \file{input.dat} and the
default output file name is \file{output.dat}.  So that you can give your
files meaningful names, these defaults can be changed by specifying
the input file name and output file name on the the command line.
The syntax is:

{\tt psi3 input-name output-name}

PSI3 is a modular program, with each module performing specific tasks
and computations.  Which modules are run for a particular computation
depends on the type of computation and the particular keywords specified
in the input file.  All keywords in PSI3 use the structure {\tt keyword =
value}, where values may be strings, booleans, integers, or real numbers.
If the value is a string which contains a special character (such as a
space or a dash) you must enclose the string in double quotation marks.
You can give keywords in the input file for specific modules; however,
in the first few examples, we will place all our keywords in one section
of our input file called {\tt psi}.  Generally, every module you run
during your computation will read the keywords in {\tt psi}, so you
can place all your keywords in this section if you choose to do so.

\subsection{Running a basic SCF calculation}
In our first example, we will consider a Hartree-Fock SCF computation
for the water molecule using a cc-pVDZ basis set.  We will specify the
geometry of our water molecule using a standard z-matrix.

\begin{verbatim}
psi:(
 label = "cc-pVDZ SCF H2O"
 jobtype = sp
 wfn = scf
 reference = rhf
 basis = "cc-pVDZ"
 zmat = (
   o
   h 1 0.957
   h 1 0.957 2 104.5
  )
 )

\end{verbatim}

In each computation, you can specify the type of wavefunction (keyword
{\tt wfn}), the reference wavefunction for post-Hartree-Fock computations
(keyword {\tt reference}), and the type of computation you want to
perform (keyword {\tt jobtype}).  In the example above, we used a
restricted Hartree-Fock (RHF) reference in an SCF computation of a
single-point energy.  To change the level of electron correlation, one
would specify a different wavefunction type using the keyword {\tt wfn}.
In the example above, to perform an MP2 computation, simply set {\tt
wfn = mp2}.

\subsection{Geometry Optimization and Vibrational Frequency Analysis}
The above example was a simple single-point energy computation.
To perform a different type of computation, change the keyword {\tt
jobtype}.  In the example below, we will set up
a CCSD geometry optimization.  To illustrate a more flexible z-matrix
input, we will now define variables for the bond length and bond angle
(in the {\tt zvars} section).

\begin{verbatim}
% 6-31G** H2O Test optimization calculation

psi: (
  label = "6-31G** SCF H2O"
  jobtype = opt
  wfn = ccsd
  reference = rhf
  dertype = first
  basis = "6-31G**"
  zmat = (
    o
    h 1 roh
    h 1 roh 2 ahoh
  )
  zvars = (
    roh     0.96031231
    ahoh  104.09437511
  )
)
\end{verbatim}

Once you have optimized the geometry of a molecule, you might wish to
perform a frequency analysis to determine the nature of the stationary
point.  To do this, change the value of {\tt jobtype} to {\tt freq}.
For an SCF frequeny calculation, you would also set {\tt dertype =
second} to compute the second derivatives analytically.  Unfortunately,
analytical second derivitives are not available in \PSIthree\ for
wavefunctions beyond SCF, so instead use the highest order analytical
derivitives that are available for the type of wavefunction you
have chosen.  This information is given in Table \ref{table:methods}.
For our CCSD example, the highest-order derivitives available are first,
so {\tt dertype = first}.

\begin{verbatim}
% 6-31G** H2O Test computation of frequencies

psi: (
  label = "6-31G** SCF H2O"
  jobtype = freq
  wfn = ccsd
  reference = rhf
  dertype = first
  basis = "6-31G**"
  zmat = (
    o
    h 1 roh
    h 1 roh 2 ahoh
  )
  zvars = (
    roh     0.96031231
    ahoh  104.09437511
  )
)
\end{verbatim}

\subsection{More Advanced Input Options}
If you wish to add comments to your input file, you can start any line
with \% and the line will be a comment line.  This can make the input
file easier to understand because you can provide explainations about
each keyword.  Another way to make the input file more organized is
to seperate it into sections that correspond to particular modules
the calculation will use.  This can be particularly helpful for more
complicated computations which can utilize many of keywords.  In the example
below, a CCSD(T) computation for the BH molecule is performed using a
cc-pVDZ basis set.  The keywords are divided into sections and several
new keywords are introduced, including ones to specify symmetry and
orbital occupations.  Orbitial occupations are specified by
a list of integers enclosed in parentheses.  These integers give the
number of orbitials which belong to each irreducible representation in
the point group.  The ordering of the irreps are those given by Cotton
in {\em Chemical Applications of Group Theory}.  In this example,
comment lines will be included to explain the new keywords used.

\begin{verbatim}
psi: (
  wfn = ccsd_t
  reference = rhf
)

default: (
  label = "BH cc-pVDZ CCSD(T)"

% Allocating memory for the calculation
  memory = (600.0 MB)

% charge and multiplicity (2S+1) default to values of 0 and 1, respectively
  charge = 0
  multp = 1

% The program will generally guess the symmetry of the molecule, but
% it can be overridden.  Here we specify C2V because only D2H and its
% subgroups can be used by the program.
  symmetry = c2v

% Number of doubly-occupied orbitals per irrep can be specified manually
% if desired
  docc = (3 0 0 0)

% Freeze the 1A1 orbital (Boron 1s-like) in the CCSD(T) computation
  frozen_docc = (1 0 0 0)
)

% The input section contains information about the molecule and the basis
% set.  The geometry here is specified by cartesian coordinates.
input: (
  basis = "cc-pVDZ"
  units = angstroms
  geometry = (
    ( b      0.0000        0.0000        0.0000)
    ( h      0.0000        0.0000        0.8000)
      )
  origin = (0.0 0.0 0.0)
)
% The modular input structure lets you specify convergence criteria for
% each part of the computation separately
scf: (
  maxiter = 100
  convergence = 11
)
\end{verbatim}

The final example of this tutorial demonstrates an example of a
complete-active-space self-consistent-field (CASSCF)
computation.  CAS computations require specification of several additional
keywords because you must specify which orbitals you wish to be in the
active space.  The notation and ordering for specifying CAS orbitals is the
same as for occupied orbitals.

\begin{verbatim}

% 6-31G** H2O Test CASSCF Energy Point

psi: (
  label = "6-31G** CASSCF H2O"
  jobtype = sp
  wfn = casscf
  reference = rhf
% The restricted_docc orbitals are those which are optimized, but are not
% in the active space.
  restricted_docc = (1 0 0 0)

% The active space orbitals; here, the valence orbitals are chosen
  active          = (3 0 1 2)

  basis = "6-31G**"
  zmat = (
    o
    h 1 1.00
    h 1 1.00 2 103.1
  )
)
\end{verbatim}



\section{\PSIthree\ Input Files} \label{input}
%\section{\PSIthree\ Input Files} \label{input}

\subsection{Syntax} \label{syntax}
\PSIthree\ input files are case-insensitive and free-format, with
a grammar designed for maximum flexibility and relative simplicity.
Input values are assigned using the structure:
\begin{verbatim}
keyword = value
\end{verbatim}
where {\tt keyword} is the parameter chosen (e.g., {\tt convergence})
and {\tt value} has one of the following data types:
\begin{itemize}
\item string: A character sequence surrounded by double-quotes.
  Example: {\tt basis = "cc-pVDZ"}
\item integer: Any positive or negative number (or zero) with no
  decimal point.  Example: {\tt maxiter = 100}
\item real: Any floating-point number.  Example: {\tt omega = 0.077357}
\item boolean: {\tt true}, {\tt false}, {\tt yes}, {\tt no}, {\tt 1},
  {\tt 0}.
\item array: a parenthetical list of values of the above data types.
  Example: {\tt docc = (3 0 1 1)}.  
\end{itemize}
Note that the input parsing system is general enough to allow
multidimensional arrays, with elements of more than one data type.  A
good example is the z-matrix keyword:
\begin{verbatim}
zmat = (
  (O)
  (H 1 r)
  (H 1 r 2 a)
)
\end{verbatim}
For z-matrices, z-matrix variables, and Cartesian coordinates, 
it is also possible to discard the inner parentheses.
The following is equivalent in this case:
\begin{verbatim}
zmat = (
  O
  H 1 r
  H 1 r 2 a
)
\end{verbatim}

Keywords must grouped together in blocks, based on the module or
modules that require them.  The default block is labelled {\tt psi:},
and most users will require only a {\tt psi:} block when using
\PSIthree.  For example, the following is a simple input file for a
single-point CCSD energy calculation on H$_2$O:
\begin{verbatim}
psi: (
  label = "6-31G**/CCSD H2O"
  wfn = ccsd
  reference = rhf
  jobtype = sp
  basis = "6-31G**"
  zmat = (
     O
     H 1 r 
     H 1 r 2 a 
  )
  zvars = (
     r 1.0 
     a 104.5 
  )
)
\end{verbatim}
In this example, the {\tt psi:} identifier collects all the keywords
(of varying types) together.  Every \PSIthree\ module will have access
to every keyword in the {\tt psi:} block by default.  One may use
other identifiers (e.g., {\tt ccenergy:}) to separate certain keywords
to be used only by selected modules.  For example, consider the
keyword {\tt convergence}, which is used by several \PSIthree\ modules
to determine the convergence criteria for constructing various types
of wave functions.  If one wanted to use a high convergence cutoff for the
\PSIthree\ SCF module but a lower cutoff for the coupled cluster
module, one could modify the above input:
\begin{verbatim}
psi: (
  ...
  convergence = 7
)
scf:convergence = 12
\end{verbatim}
Note that, since we have only one keyword associated with the {\tt
  scf:} block, we do not need to enclose it parentheses.

Some additional aspects of the \PSIthree\ grammar to keep in mind:
\begin{itemize}
\item The ``\%'' character denotes a comment line, i.e. any
  information following the ``\%'' up to the next linebreak is ignored
  by the program.
\item Anything in between double quotes (i.e. strings) is case-sensitive.
\item Multiple spaces are treated as a single space.
\end{itemize}

\subsection{Specifying the Type of Computation}
The most important keywords in a \PSIthree\ input file are those which
tell the program what type of computation are to be performed.  
They \keyword{jobtype} keyword tells the \PSIdriver\ program whether
this is a single-point computation, a geometry optimization, a 
vibrational frequency calculation, etc.  The \keyword{reference} 
keyword specifies whether an RHF, ROHF, UHF, etc., reference is
to be used for the SCF wavefunction.  The \keyword{wfn} specifies 
what theoretical method is to be used, either SCF, determinant-based
CI, coupled-cluster, etc.  Also of critical importance are the charge
and multiplicity of the molecule, the molecular geometry, and the
basis set to be used.  The latter two topics are discussed below
in sections \ref{geom-spec} and \ref{basis-spec}.
General keywords determining the general type of computation to be performed 
are described below.

\begin{description}
\item[LABEL = string]\mbox{}\\
This is a character string to be included in the output to help keep track
of what computation has been run.  It is not otherwise used by the program.
There is no default.
\item[JOBTYPE = string]\mbox{}\\
This tells the program whether to run a single-point energy calculation
(SP), a geometry optimization (OPT), a series of calculations at 
different displaced geometries (DISP), a frequency calculation (FREQ),
frequencies only for symmetric vibrational modes (SYMM\_FREQ), 
a Diagonal Born-Oppenheimer Correction (DBOC) energy computation,
or certain response properties (RESPONSE).
The default is SP.
\item[WFN = string]\mbox{}\\
This specifies the wavefunction type.  Possible values are:\\ 
SCF, MP2, MP2R12, CIS, DETCI, CASSCF, RASSCF, CCSD, CCSD\_T, BCCD, BCCD\_T,\\
EOM\_CCSD. 
\item[REFERENCE = string]\mbox{}\\
This specifies the type of SCF calculation one wants to do.  It
can be one of RHF (for a closed  shell  singlet), ROHF (for
a restricted open shell calculation), UHF (for an unrestricted
open shell calculation), or TWOCON (for a two configuration
singlet).  The default is RHF.
\item[MULTP = integer]\mbox{}\\
Specifies the multiplicity of the molecule, i.e., 2S+1.  Default
is 1 (singlet).
\item[CHARGE = integer]\mbox{}\\
Specifies the charge of the molecule.  Default is 0.
\item[DERTYPE = string]\mbox{}\\
This specifies the order of the derivative that is to be obtained.
The default is NONE (energy only).
\item[DOCC = integer vector]\mbox{}\\
This gives the number of doubly occupied orbitals in each irreducible
representation.  There is no default.  If this is not given, 
\PSIcscf\  will attempt to guess at the occupations.
\item[SOCC = integer vector]\mbox{} \\
This gives the number of singly occupied orbitals in each irreducible 
representation. There is no default.  If this is not given,
\PSIcscf\ will attempt to guess at the occupations.
\item[FREEZE\_CORE = string]\mbox{} \\
\PSIthree\ can automatically freeze core orbitals. Core orbitals are
defined as follows:  
\begin{verbatim}
 H-Be  no core 
 B-Ne  1s 
Na-Ar  small: 1s2s
       large: 1s2s2p
\end{verbatim}
YES or TRUE will freeze the core orbitals, SMALL or LARGE are for elements 
Na-Ar. The default is NO or FALSE. Always check to make sure that the 
occupations are correct!
\end{description}

\subsection{Geometry Specification} \label{geom-spec}
The molecular geometry may be specified using either Cartesian a
Z-matrix coordinates.  Cartesian coordinates are specified via the
keyword \keyword{geometry}:
\begin{verbatim}
  geometry = (
     atomname1 x1 y1 z1 
     atomname2 x2 y2 z2 
     atomname3 x3 y3 z3 
             ...
     atomnameN xN yN zN 
  )
\end{verbatim}
where \keyword{atomname$i$} can take the following values:
\begin{itemize}
\item The element symbol: H, He, Li, Be, B, etc.
\item The full element name: hydrogen, helium, lithium, etc.
\item As a {\em ghost} atom with the symbol, G, or name, ghost. A
ghost atom has a formal charge 0.0, and can be useful to specify the
location of the off-nucleus basis functions.
\item As a {\em dummy} atom with the symbol, X.  Dummy atoms can be
useful only to specify Z-matrix coordinates of proper symmetry or
which contain linear fragments.
\end{itemize}
Hence the following two examples are equivalent to one another:
\begin{verbatim}
  geometry = (
     H 0.0 0.0 0.0 
     f 1.0 0.0 0.0 
     Li 3.0 0.0 0.0 
     BE 6.0 0.0 0.0 
  )
\end{verbatim}
\begin{verbatim}
  geometry = (
     hydrogen  0.0 0.0 0.0 
     FLUORINE  1.0 0.0 0.0 
     Lithium   3.0 0.0 0.0 
     beryllium 6.0 0.0 0.0 
  )
\end{verbatim}
It is also possible to include an inner set 
of parentheses around each line containing {\tt atomname1 x1 y1 z1}.

The keyword \keyword{units} specifies the units for the coordinates:
\begin{itemize}
\item \keyword{units = angstrom} -- angstroms (\AA), default;
\item \keyword{units = bohr} -- atomic units (Bohr);
\end{itemize}

\noindent
Z-matrix coordinates are specified using the keyword \keyword{zmat}:
\begin{verbatim}
  zmat = (
     atomname1
     atomname2 ref21 bond_dist2
     atomname3 ref31 bond_dist3 ref32 bond_angle3 
     atomname4 ref41 bond_dist4 ref42 bond_angle4 ref43 tors_angle4 
     atomname5 ref51 bond_dist5 ref52 bond_angle5 ref53 tors_angle5 
                             ...                
     atomnameN refN1 bond_distN refN2 bond_angleN refN3 tors_angleN 
  )
\end{verbatim}
where
\begin{itemize}
\item \keyword{bond\_dist$i$} is the distance (in units specified by
keyword \keyword{units}) from nucleus number $i$ to
nucleus number \keyword{ref$i$1}. The units 
\item \keyword{bond\_angle$i$} is the angle formed by nuclei $i$,
\keyword{ref$i$1}, and \keyword{ref$i$2};
\item \keyword{tors\_angle$i$} is the torsion angle formed by nuclei $i$,
\keyword{ref$i$1}, \keyword{ref$i$2}, and \keyword{ref$i$3};
\end{itemize}
%%
%% I'm commenting this part out, since the redundant internal coordinate
%% structure implemented by RAK in 08/03 takes care of many dummy-atom
%% problems.  If user's really want z-matrix coords, we may have to deal
%% with this again.
%%   -TDC, 08/31/03
%Some care has to be taken when constructing a Z-matrix for a molecule
%which contains linear fragments. For example, an appropriate Z-matrix
%for a linear conformation of HNCO must include dummy atoms: The first
%three atoms (HNC) can be specified as is, but the fourth atom (O)
%poses a problem -- the torsional angle cannot be defined with respect
%to the linear HNC fragment. The solution is to add 2 dummy atoms to
%the definition:
%\begin{verbatim}
%  zmat = (
%    h
%    n 1 1.012
%    x 2 1.000 1  90.0
%    c 2 1.234 3  90.0 1 180.0
%    x 4 1.000 2  90.0 3 180.0
%    o 4 1.114 5  90.0 2 180.0
%  )
%\end{verbatim}
%Alternatively, one could use, for example, only a single dummy atom above the
%nitrogen and specify ``bond lengths'' relative to the latter.

\subsection{Molecular Symmetry} \label{symm-spec}
\PSIthree\ can determine automatically the largest Abelian point group
for a valid framework of centers (including ghost atoms, but dummy
atoms are ignored).  It will then use the symmetry properties of the
system in computing the energy, forces, and other properties.
However, in certain instances it is desirable to use less than the
full symmetry of the molecule. The keyword \keyword{subgroup} is used
to specify a subgroup of the full molecular point group. The allowed
values are \keyword{c2v}, \keyword{c2h}, \keyword{d2}, \keyword{c2},
\keyword{cs}, \keyword{ci}, and \keyword{c1}. For certain combinations
of a group and its subgroup there is no unique way to determine which
subgroup is implied. For example, $D_{\rm 2h}$ has 3 non-equivalent
$C_{\rm 2v}$ subgroups, e.g. $C_{\rm 2v}(X)$ consists of symmetry
operations $\hat{E}$, $\hat{C}_2(x)$, $\hat{\sigma}_{xy}$, and
$\hat{\sigma_{xz}}$.  To specify such subgroups precisely one has to
use the keyword \keyword{unique\_axis}.  For example, the following
input will specify the $C_{\rm 2v}(X)$ subgroup of $D_{\rm 2h}$ to be
the computational point group:
\begin{verbatim}
  psi: (
    ...
    geometry = (
         ...
    )
    units = angstrom
    subgroup = c2v
    unique_axis = x
  )
\end{verbatim}

\begin{table}[h]
%\caption{Standard Cotton Ordering in \PSIthree}
\begin{center}
\begin{tabular}{ll}
\hline
\hline
Point Group & Cotton Ordering of Irreps \\
\hline
C$_1$       & A                                    \\
C$_i$       & A$_g$ A$_u$                          \\
C$_2$       & A B                                  \\
C$_s$       & A' A''                               \\
C$_{2h}$    & A$_g$ B$_g$ A$_u$ B$_u$              \\
C$_{2v}$    & A$_1$ A$_2$ B$_1$ B$_2$              \\
D$_2$       & A B$_1$ B$_2$ B$_3$                  \\
D$_{2h}$    & A$_g$ B$_{1g}$ B$_{2g}$ B$_{3g}$ A$_u$ B$_{1u}$ B$_{2u}$ B$_{3u}$ \\
\hline
\hline
\end{tabular}
\end{center}
\end{table}

\subsection{Specifying Scratch Disk Usage in \PSIthree} \label{scratchfiles}

Depending on the calculation, the \PSIthree\ package often requires
substantial temporary disk storage for integrals, wave function ampltiudes,
etc.  By default, \PSIthree\ will write all such datafiles to {\tt /tmp}
(except for the checkpoint file, which is written to {\tt ./} by default).
However, to allow for various customized arrangements of scratch disks,
the \PSIthree\ {\tt files:} block gives the user considerable control
over how temporary files are organized, including file names, scratch
directories, and the ability to ``stripe'' files over several disks (much
like RAID0 systems).  This section of keywords is normally placed within
the {\tt psi:} section of input, but may be used for specific \PSIthree\
modules, just like other keywords.

For example, if the user is working with \PSIthree\ on a computer
system with only one scratch disk (mounted at, e.g., {\tt /scr}), one
could identify the disk in the input file as follows:
\begin{verbatim}
psi: (
  ...
  files: (
    default: (
      nvolume = 1
      volume1 = "/scr/"
    )
  )
)
\end{verbatim}
The {\tt nvolume} keyword indicates the number of scratch
directories/disks to be used to stripe files, and each of these is
specified by a corresponding {\tt volumen} keyword.  (NB: the trailing
slash ``/'' is essential in the directoy name.)  Thus, in the above
example, all temporary storage files generated by the various
\PSIthree\ modules would automatically be placed in the {\tt /scr}
directory.  

By default, the scratch files are given the prefix ``{\tt psi}'', and
named ``{\tt psi.nnn}'', where {\tt nnn} is a number used by the
\PSIthree\ modules.  The user can select a different prefix by
specifying it in the input file with the {\tt name} keyword:
\begin{verbatim}
psi: (
  ...
  files: (
    default: (
      name = "H2O"
      nvolume = 1
      volume1 = "/scr/"
    )
  )
)
\end{verbatim}
The {\tt name} keyword allows the user to store data associated with
multiple calculations in the same scratch area.  Alternatively, one
may specify the filename prefix on the command-line of the {\tt psi3}
driver program (or any \PSIthree\ module) with the {\tt -p} argument:
\begin{verbatim}
psi3 -p H2O
\end{verbatim}

If the user has multiple scratch areas available, \PSIthree\ files may
be automatically split (evenly) across them:
\begin{verbatim}
psi: (
  ...
  files: (
    default: (
      nvolume = 3
      volume1 = "/scr1/"
      volume2 = "/scr2/"
      volume3 = "/scr3/"
    )
  )
)
\end{verbatim}
In this case, each \PSIthree\ datafile will be written in chunks (65
kB each) to three separate files, e.g., {\tt /scr1/psi.72}, {\tt
/scr2/psi.72}, and {\tt /scr3/psi.72}.  The maximum number of volumes
allowed for striping files is eight (8), though this may be easily
extended in the \PSIthree\ I/O code, if necessary.

The format of the {\tt files} section of input also allows the user to
place selected files in alternative directories, such as the current
working directory.  This feature is especially important if some of
the data need to be retained between calculations.  For example, the
following {\tt files:} section will put {\tt file32} (the \PSIthree\
checkpoint file) into the working directory, but all scratch files
into the temporary areas:
\begin{verbatim}
psi: (
  ...
  files: (
    default: (
      nvolume = 3
      volume1 = "/scr1/"
      volume2 = "/scr2/"
      volume3 = "/scr3/"
    )
    file32: ( nvolume = 1  volume1 = "./" )
  )
)
\end{verbatim}

\subsection{The {\tt .psirc} File} \label{psirc}

Users of \PSIthree\ often find that they wish to use certain keywords
or input sections in every calculation they run, especially those
keywords associated with the {\tt files:} section.  The {\tt .psirc}
file, which is kept in the user's {\tt \$HOME} directory, helps to
avoid repetition of keywords whose defaults are essentially user- or
system-specific.  A typical {\tt .psirc} file would look like:
\begin{verbatim}
psi: (
  files: (
    default: (
      nvolume=3
      volume1 = "/tmp1/mylogin/"
      volume2 = "/tmp2/mylogin/"
      volume3 = "/tmp3/mylogin/"
    )
    file32: (nvolume=1 volume1 = "./")
  )
)
\end{verbatim}

\subsection{Specifying Basis Sets} \label{basis-spec}

\PSIthree\ uses basis sets comprised of Cartesian or spherical harmonic
Gaussian functions. A basis set is identified by a string, enclosed
in double quotes. Currently, there exist three ways to specify which
basis sets to use for which atoms:
\begin{itemize}
\item \keyword{basis = string} -- all atoms use basis set type.
\item \keyword{basis = (string1 string2 string3 ... stringN)} -- 
\keyword{string {\em i}}
specifies the basis set for atom {\em i}. Thus, the number of strings
in the \keyword{basis} vector has to be the same as the number of
atoms (including ghost atoms but excluding dummy atoms). Another
restriction is that symmetry equivalent atoms should have same basis
sets, otherwise \PSIinput\ will use the string provided for the
so-called unique atom out of the set of symmetry equivalent ones.
\item 
\begin{verbatim}
  basis = (
    (element1 string1)
    (element2 string2)
            ...
    (elementN stringN)
  )
\end{verbatim}
\keyword{string {\em i}} specifies the basis set for chemical element 
\keyword{element {\em i}}.
\end{itemize}

\subsubsection{Default Basis Sets}

\PSIthree\ default basis sets are located in \pbasisdat\ which may be
found by default in {\tt \$psipath/share}. Tables
\ref{table:poplebasis}, \ref{table:dunningbasis},
\ref{table:wachtersbasis}, and \ref{table:ccbasis} list basis sets
pre-defined in \pbasisdat.

The predefined basis sets use either spherical harmonics or Cartesian
Gaussians, which is determined by the authors of the basis.
Currently \PSIthree\ cannot handle basis sets that consist
of a mix of Cartesian and spherical harmonics Gaussians.
Therefore there may be combinations of basis sets that are forbidden,
e.g. {\tt cc-pVTZ} and {\tt 6-31G**}.
In such case one can override the predetermined choice
of the type of the Gaussians by specifying the \keyword{puream}
keyword. It takes two values, {\tt true} or {\tt false},
for spherical harmonics and Cartesian Gaussians, respectively.

\begin{table}[p]
\caption{Pople-type basis sets available in \PSIthree}
\label{table:poplebasis}
\begin{center}
\begin{tabular}{|l|l|l|}
\hline
\hline
Basis Set 		&Atoms   	&Aliases\\ 
\hline
	STO-3G			& H-Ar			&\\
	3-21G			& H-Ar			&\\
	6-31G			& H-Ar, K, Ca, Cu	&\\
        6-31G*                  & H-Ar, K, Ca, Cu       &6-31G(d)\\
        6-31+G*                 & H-Ar                  &6-31+G(d)\\
        6-31G**                 & H-Ar, K, Ca, Cu       &6-31G(d,p)\\
	6-311G			& H-Ar			&\\
	6-311G*                 & H-Ar                  &6-311G(d)\\
        6-311+G*                & H-Ne                  &6-311+G(d)\\
	6-311G**                & H-Ar                  &6-311G(d,p)\\
        6-311G(2df,2pd)         & H-Ne                  &\\
	6-311++G**		& H, B-Ar		&6-311++G(d,p)\\
        6-311G(2d,2p)           & H-Ar                  &\\
        6-311++G(2d,2p)         & H-Ar                  &\\
        6-311++G(3df,3pd)       & H-Ar                  &\\
\hline
\hline
\end{tabular}
\end{center}
\end{table}

\begin{table}[tbp]
\caption{Huzinaga-Dunning basis sets available in \PSIthree}
\label{table:dunningbasis}
\begin{center}
\begin{tabular}{|l|l|}
\hline
\hline
Basis Set 		&Atoms   	\\
\hline
	(4S/2S)			& H		\\
	(9S5P/4S2P)		& B-F			\\
	(11S7P/6S4P)		& Al-Cl			\\
	DZ			& H, Li, B-F, Al-Cl		\\
	DZP			& H, Li, Be, B-F, Na, Al-Cl	\\
	DZ-DIF			& H, B-F, Al-Cl		\\
	DZP-DIF			& H, B-F, Al-Cl		\\
	TZ2P			& H, B-F, Al-Cl		\\
	TZ2PD			& H			\\
	TZ2PF			& H, B-F, Al-Cl		\\
	TZ-DIF			& H, B-F, Al-Cl		\\ 	
	TZ2P-DIF		& H, B-F, Al-Cl		\\
	TZ2PD-DIF		& H			\\
	TZ2PF-DIF		& H, B-F, Al-Cl		\\		
\hline
\hline
\end{tabular}
\end{center}
\end{table}

\begin{table}[tbp]
\caption{Wachters basis sets available in \PSIthree}
\label{table:wachtersbasis}
\begin{center}
\begin{tabular}{|l|l|}
\hline
\hline
Basis Set 		&Atoms   	\\ 
\hline
	WACHTERS		& K, Sc-Cu			\\
	WACHTERS-F		& Sc-Cu			\\
\hline
\hline
\end{tabular}
\end{center}
\end{table}

\begin{table}[tbp]
\caption{Correlation-consistent basis sets available in \PSIthree}
\label{table:ccbasis}
\begin{center}
\begin{tabular}{|l|l|l|}
\hline
\hline
Basis Set 		&Atoms   	&Aliases\\ 
\textbf{ (N = D,T,Q,5,6)}	&			&	\\
\hline
	cc-pVNZ			& H-Ar			&CC-PVNZ\\
	cc-pV(N+D)Z		& Al-Ar			&CC-PV(N+D)Z\\
        cc-pCVNZ                & B-Ne                  &CC-PCVNZ\\
	aug-cc-pVNZ		& H-He, B-Ne, Al-Ar	&AUG-CC-PVNZ\\
	aug-cc-pV(N+D)Z		& Al-Ar			&AUG-CC-PCV(N+d)Z\\
	aug-cc-pCVNZ    	& B-F (N${<}$6)		&AUG-CC-PCVNZ\\
	d-aug-cc-pVNZ		& H			&\\
	pV7Z\footnote{testa}	& H, C, N, O, F, S	&PV7Z\\
	cc-pV7Z\footnote{testb}	& H, C, N, O, F, S	&CC-PV7Z\\
	aug-pV7Z\footnote{testc}     & H, C, N, O, F, S	&AUG-PV7Z\\
	aug-cc-pV7Z\footnote{testd}  & H, N, O, F            &AUG-CC-PV7Z\\
\hline
\hline
\end{tabular}
\end{center}
\end{table}

\subsubsection{Custom Basis Sets} \label{custom-basis}

If the basis set you desire is not already defined in \PSIthree, a
custom set may be used by specifying its exponents and contraction
coefficients (either in the input file or another file named {\tt
basis.dat}.) A contracted Cartesian Gaussian-type orbital
\begin{equation}
\phi_{\rm CGTO} =  x^ly^mz^n\sum_i^N C_i \exp(-\alpha_i[x^2+y^2+z^2])
\end{equation}
where
\begin{equation}
L = l+m+n
\end{equation}
is written as
\begin{verbatim}
basis: (
  ATOM_NAME: "BASIS_SET_LABEL" = (
    (L (C1  alpha1)
       (C2  alpha2)
       (C3  alpha3)
       ...
       (CN  alpha4))   
    )
  )
\end{verbatim}

One must further specify whether Cartesian or spherical harmonics
Gaussians are to be used. One can specify that in two ways:
\begin{itemize}
\item It can be done on a basis
by basis case, such as
\begin{verbatim}
basis: (
  "BASIS_SET_LABEL1":puream = true
  "BASIS_SET_LABEL2":puream = false
  "BASIS_SET_LABEL3":puream = true
  ....
)
\end{verbatim}
By default, if \keyword{puream} is not given for a basis,
then Cartesian Gaussians will be used.
\item
The choice between Cartesian or spherical harmonics Gaussian
can be made globally by specifying \keyword{puream} keyword
in the standard input section, e.g.
\begin{verbatim}
psi: (
  ...
  puream = true
  ...
)
\end{verbatim}
\end{itemize}
Note that currently \PSIthree\ cannot handle basis sets that consist
of a mix of Cartesian and spherical harmonics Gaussians.

Note that the basis set must be given in a separate {\tt basis:}
section of input, outside all other sections (including {\tt psi:}).
For example, the \PSIthree\ DZP basis set for carbon could be
specified as:
\begin{verbatim}
basis: (
  carbon: "DZP" = (
    (S (   4232.6100      0.002029) 
       (    634.8820      0.015535)
       (    146.0970      0.075411)
       (     42.4974      0.257121)
       (     14.1892      0.596555) 
       (      1.9666      0.242517))
    (S (      5.1477      1.0))
    (S (      0.4962      1.0))
    (S (      0.1533      1.0))
    (P (     18.1557      0.018534)
       (      3.9864      0.115442)
       (      1.1429      0.386206)
       (      0.3594      0.640089))
    (P (      0.1146      1.0))
    (D (      0.75        1.0))
  )
)
\end{verbatim}

Here are a couple of additional points that may be useful when
specifying customized basis sets:
\begin{itemize}
\item Normally the {\tt basis.dat} file is placed in the same
directory as the main input file, but it may also be placed in a
global location specified by the keyword {\tt basisfile}:
\begin{verbatim}
  basisfile = "/home/users/tool/chem/h2o/mybasis.in"
\end{verbatim}
\item To scale a basis set, a scale factor may be added as the last item
in the specification of each contracted Gaussian function.  For
example, to scale the S functions in a 6-31G** basis for hydrogen,
one would use the following
\begin{verbatim}
  hydrogen: "6-31G**" =
      ( (S (    18.73113696     0.03349460)
           (     2.82539437     0.23472695)
           (     0.64012169     0.81375733) 1.2 )
        (S (     0.16127776     1.00000000) 1.2 )
        (P (     1.10000000     1.00000000))
       )
\end{verbatim}
In this example, both contracted S functions have their exponents
scaled by a factor of (1.2)$^2$ = 1.44.  The output file should show
the exponents after scaling.
\end{itemize}

\subsubsection{Automated Conversion of Basis Sets}

The \PSIthree\ package is distributed with a Perl-based utility, named {\tt
g94\_2\_PSI3}, which will convert basis sets from the Gaussian ('94 or later)
format to \PSIthree\ format automatically.  This utility is especially useful
for basis sets downloaded from the EMSL database at \htmladdnormallink{{\tt
http://www.emsl.pnl.gov/forms/basisform.html}}{http://www.emsl.pnl.gov/forms/basisform.html}.
To use this utility, save the desired basis set to a file (e.g., {\tt
g94\_basis.dat}) in the Gaussian format.  Then execute:
\begin{verbatim}
g94_2_PSI3 < g94_basis.dat > basis.dat
\end{verbatim}
You may either incorporate the results from the {\tt basis.dat} file into
your input file as described above, or place the results into a global {\tt
basis.dat} file.  Be sure to surround the basis-set definition with the
{\tt basis:()} keyword (as shown in the above examples) or input parsing
errors will result.


\section{Overview of \PSIthree\ Computations} \label{overview}

\subsection{Hartree-Fock SCF}
\subsection{MP2 and MP2-R12}
\subsection{Coupled Cluster}
\subsection{Configuration Interaction}
\subsection{Complete Active Space SCF (CASSCF)}
\subsection{Excited State Methods: CIS, RPA, EOM-CCSD}
\subsection{Geometry Optimizations and {\tt optking}}
\subsection{Vibrational Frequency Analyses}
\subsection{One-Electron Properties}
\subsection{User-Specified Basis Sets}

\newpage
\appendix
\section{\PSIthree\ Reference}\label{PSI_Reference}
%T. Daniel Crawford, C. David Sherrill, Edward F. Valeev, Justin
T. Fermann, Rollin A. King, Matthew L. Leininger, Shawn T. Brown,
Curtis L. Janssen, Edward T. Seidl, Joseph P. Kenny, and Wesley D. Allen,
{\em J. Comput. Chem.}, in press.


%\section{Sample \inputdat\ file}
\begin{verbatim}
default: (
%%%%%%%%%%%%%%%%%%%%%%%%%%%
% Test input.dat for errors
%%%%%%%%%%%%%%%%%%%%%%%%%%%
% check = true
  check = false

%%%%%%%%%%%%%%%%%%%%%%%%%%%
% System parameters and job title
%%%%%%%%%%%%%%%%%%%%%%%%%%%
  memory = (50.0 MB)
  label = "CH3 cc-pVTZ SCF"

%%%%%%%%%%%%%%%%%%%%%%%%%%%
% Choose wavefunction
%%%%%%%%%%%%%%%%%%%%%%%%%%%
  wfn = scf
% wfn = mp2
% wfn = detci
% wfn = ccsd

%%%%%%%%%%%%%%%%%%%%%%%%%%%
% Derivative level
%%%%%%%%%%%%%%%%%%%%%%%%%%%
% dertype = none
  dertype = first


%%%%%%%%%%%%%%%%%%%%%%%%%%%
% Electronic structure info
%%%%%%%%%%%%%%%%%%%%%%%%%%%
% reference = rhf
  reference = rohf
% reference = uhf
% reference = twocon

%%%%%%%%%%%%%%%%%%%%%%%%%%%%%%%%
% Optimize geometry 
%%%%%%%%%%%%%%%%%%%%%%%%%%%%%%%%
% opt = true
  opt = true
  nopt = 10

%%%%%%%%%%%%%%%%%%%%%%%%%%%%%%%%
% Finite difference
%%%%%%%%%%%%%%%%%%%%%%%%%%%%%%%%
% disp = true
% ndisp = 1

%%%%%%%%%%%%%%%%%%%%%%%%%%%%%%%%
% Scratch files
% note: you must change psiuser
% to your username
%%%%%%%%%%%%%%%%%%%%%%%%%%%%%%%%
  files: (
         default: (
                  name = "ch3"
                  volume1 = "/tmp1/psiuser/"
                  volume2 = "/tmp2/psiuser/"
                  volume3 = "/tmp3/psiuser/"
                  volume4 = "/tmp4/psiuser/"
                  )
         file32: ( nvolume = 1 volume1 = "./" )
         )
)

%%%%%%%%%%%%%%%%%%%%%%%%%%%%%%%%
% Section that generate file30
%%%%%%%%%%%%%%%%%%%%%%%%%%%%%%%%
input: (
  units = angstrom
  basis = ccpvtz
  zmat = (
   (x)
   (c 1 1.0)
   (h 2 0.97 1 90.0)
   (h 2 0.97 1 90.0 3  120.0)
   (h 2 0.97 1 90.0 3 -120.0)
  )
)
\end{verbatim}

\section{Sample \basisdat\ file}
\begin{verbatim}
basis: (
  NITROGEN:t2pd = (
    (S (  13520.          0.000760)
       (   1999.          0.006076)
       (    440.0         0.032847)
       (    120.9         0.132396)
       (     38.47        0.393261)
       (     13.46        0.546339))
    (S (     13.46        0.252036)
       (      4.993       0.779385))
    (S (      1.569       1.000000))
    (S (      0.5800      1.000000))
    (S (      0.1923      1.000000))
    (S (      0.06742     1.000000))
    (P (     35.91        0.016916)
       (      8.480       0.102200)
       (      2.706       0.338134)
       (      0.9921      0.669281))
    (P (      0.3727      1.000000))
    (P (      0.1346      1.000000))
    (P (      0.04959     1.000000))
    (D (      1.60        1.000000))
    (D (      0.40        1.000000))
                    )

 HYDROGEN:t2pd = (
    (S (       33.64             0.025374)
       (        5.058            0.189684)
       (        1.147            0.852933))
    (S (        0.3211           1.0))
    (S (        0.1013           1.0))
    (S (        0.03016          1.0))
    (P (        1.50             1.0))
    (P (        0.375            1.0))
                   )
      )
\end{verbatim}


%%%%%%%%%%%%%%%%%%%%%%%
% References
%%%%%%%%%%%%%%%%%%%%%%%
\bibliographystyle{prsty}
\bibliography{bibliography}


\end{document}

